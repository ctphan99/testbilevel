\appendix
\section*{Appendix}

\section{Notation}\label{sec:appendix_notation}
We use $\langle \cdot{}, {}\cdot\rangle$ to denote inner products and $\|{}\cdot{}\|$ for the Euclidean norm. Unless transposed, all vectors are column vectors. For $f:\reals^{d_2}\to\reals^{d_1}$ its Jacobian with respect to $x\in \reals^{d_2}$ is 
$\nabla f\in \reals^{d_1 \times d_2}$.  For $f:\reals^d\to\reals$, we overload  $\nabla f$ to refer to its gradient (the transposed Jacobian), a column vector. We use 
$\nabla_x$ to denote partial derivatives with respect to $x$.  
% \pswt{notation for  partial derivatives, higher order derivatives, etc.}

A function $f:\reals^n\to\reals^m$ is $L$-Lipschitz if for any $x,y$, we have $\|f(x) - f(y)\|\leq L \|x-y\|$.
A differentiable function $f:\reals^n\to\reals$ is convex if for any $x, y\in \reals^n$ we have $f(y)\geq f(x) + \nabla f(x)^\top (y-x)$; 
it is 
 $\mu$-strongly convex
 if $f - \tfrac{\mu}{2}\|{}\cdot{}\|^2$ is convex;
it is  $\beta$-smooth
if
$\nabla f$ is $\beta$-Lipschitz.

For a Lipschitz function $f$, a point $x$ is $(\delta, \epsilon)$-stationary if within a $\delta$-ball around $x$, there exists a convex combination of  subgradients of $f$ with norm at most $\epsilon$. 
For a differentiable function $f$, we say that $x$ is $\epsilon$-stationary if $\|\nabla f(x)\|\leq \epsilon$. 


\section{Proofs from \cref{sec:equality-bilevel}} \label{sec:appendix_linear_equality}
In this section, we provide the full proofs of 
claims for bilevel programs with linear equality constraints, as stated in \cref{sec:equality-bilevel}. We first state a few technical results using the implicit function theorem that we repeatedly invoke in our results for this setting. 

\begin{restatable}{lemma}{lemDystarDxLineq}\label{lem:dystarDxLinEq}
    Fix a point $x$. Given $y^*= \arg\min_{y: h(x,y)=0} g(x,y)$ where $g$ is strongly convex in $y$ and $\lamstar$ is the dual optimal variable for this problem, define $\Leqc(x,y,\lam)=g(x,y)+\langle\lam, h(x,y)\rangle$. Then, we have 
\[\underbrace{\begin{bmatrix}\grad^2_{yy}\Leqc(x,\ystar,\lamstar) & \nabla_y h(x,y^{*})^{\top}\\
\nabla_y h(x,y^{*}) & 0
\end{bmatrix}}_{H ~\text{for linear equality constraints}}\begin{bmatrix}\frac{d\ystar}{dx}\\
\frac{d\lamstar}{dx}
\end{bmatrix}=\begin{bmatrix}-\nabla^2_{yx}g(x,y^{*})-\nabla^2_{yx}\langle\lamstar, h(x,y^{*})\rangle\\
-\nabla_x h(x,\ystar)\end{bmatrix}.
% \numberthis\label{eq:def_matrix_H}
\]
\end{restatable}
% \lemDystarDxLineq*
\begin{proof} 
Since $g$ is strongly convex, by linear
constraint qualification, the KKT condition
is both sufficient and necessary condition for optimality.
Hence, consider the
following KKT system obtained via first order optimality of $y^*$, with dual optimal variable $\lamstar$:
\begin{align*}
    \nabla_y g(x,y^{*})+\nabla_y\langle\lamstar,  h(x,y^{*})\rangle  =0, \text{ and } h(x, \ystar)=0.\numberthis\label{eqs:kkt-lin-eq}
\end{align*}
Differentiating 
the system of equations in \cref{eqs:kkt-lin-eq} with respect to $x$ and rearranging terms in a matrix-vector format yields:
\begin{equation}\label{eqs:differentiated-kkt-lin-eq}
    \begin{aligned}
\begin{bmatrix}\nabla^2_{yy} g(x,\ystar) + \nabla^2_{yy}\langle\lamstar,   h(x,\ystar)\rangle  & \nabla_y h(x,y^{*})^{\top}\\
\nabla_y h(x,y^{*}) & 0
\end{bmatrix}\begin{bmatrix}\frac{d\ystar}{dx}\\
\frac{d\lamstar}{dx}
\end{bmatrix}=\begin{bmatrix}-\nabla^2_{yx}g(x,y^{*})-\nabla^2_{yx}\langle\lamstar, h(x,y^{*})\rangle\\
-\nabla_x h(x,\ystar)\end{bmatrix}
    \end{aligned}
\end{equation}
% \begin{equation}
% \begin{aligned}
% \nabla^2_{yy} g(x,\ystar)\frac{d\ystar}{dx}+\nabla^2_{yx} g(x,\ystar)+\lamstar \nabla^2_{yy} h(x,\ystar)\frac{d\ystar}{dx}+\lamstar \nabla^2_{yx} h(x,\ystar)+\nabla_y h(x,\ystar)\frac{d\lamstar}{dx} & =0\\
% \nabla_y h(x,\ystar)\frac{d\ystar}{dx}+\nabla_x h(x,\ystar) & =0.
% \end{aligned}
% \end{equation}
Noting that $\nabla_{yy}^{2}\Leqc(x,y,\lam)=\nabla^2_{yy} g(x,y)+\nabla^2_{yy}\langle\lam,  h(x,y)\rangle$, we can write \cref{eqs:differentiated-kkt-lin-eq} in the form shown in the lemma.
\end{proof}


\begin{restatable}{lemma}{lemLineqHinvertibility}\label{lem:non-singular-req} 
Consider the setup in \cref{{lem:dystarDxLinEq}}. The matrix $H$ defined in \cref{eq:def_matrix_H}  is invertible if
the Hessian $\grad_{yy}^{2}\Leqc(x,\ystar,\lamstar):=\nabla^2_{yy} g(x, y^*) +\nabla^2_{yy}\langle\lamstar,  h(x, y^*)\rangle$
satisfies $\nabla_{yy}^{2}\Leqc(x,\ystar,\lamstar)\succ0$ over
the tangent plane $T:=\{y:\nabla_y h(x,y^{*})y=0\}$ and $\nabla_y h$ has full
rank.
% , i.e., $\ystar$ is a non-degenerate local optimal solution. 
% As such, \pswt{move this}
% \label{lem:non-singular-req}
\end{restatable}
\begin{proof}
Let $u=[y,\lam].$ We show that $Hu=0$ implies $u=0$, which in turn implies invertibility of $H$. If $\nabla_y h(x,\ystar)y\neq0,$ then by construction of $u$ and $H$, we must also have $Hu\neq0$. Otherwise if $\nabla_y h(x,\ystar)y=0$ and $y\neq0$,
the quadratic form $u^{\top}Hu$ is positive, as seen by 
\[
u^{\top}Hu=y^{\top}\grad^2_{yy}\Leqc(x,\ystar,\lamstar)y>0,
\] where the final step is by the assumption of $\Leqc$ being positive definite over the defined tangent plane $T=\{y:\nabla_y h(x,y^{*})y=0\}$. 
If $y=0$ while $Hu=0$, then $\nabla_y h$ having full rank implies $\lam=0$. Combined with $y=0$, this means $u=0$, as required when $Hu=0$. This concludes the proof. 
\end{proof}

\begin{restatable}{corollary}{corNonSingularH}\label{cor:nonsingularH}
For \cref{prob:lin-eq} under \cref{assumption:linEq_smoothness} and \cref{assumption:eq},
% \pswt{and TBD assumption}, 
 the matrix $H$  (as defined in \cref{eq:def_matrix_H}) is non-singular. Further, there exists a finite $C_H$ such that $\|H^{-1}\|\leq C_H$. 
% and $\|\frac{d\ystar}{dx}\|\leq C_H C_g$. 
\end{restatable}
\begin{proof} Since we are assuming strong convexity of $g$, \cref{{lem:non-singular-req}} applies, yielding the claimed invertibility of $H$. Combined with the boundedness of variables $x$ (per \cref{assumption:eq}) 
% \pswt{Put the equality-specific bound here for compactness 
% of $\mathcal{X}$ 
and continuity of the inverse implies a bound on $\|H^{-1}\|$. 
\end{proof}


\subsection{Construction of the inexact gradient oracle}
We now show how to construct the inexact gradient oracle for the objective $F$ in \cref{prob:lin-eq}. As sketched in \cref{sec:equality-bilevel}, we then use this oracle in a projected gradient descent algorithm to get the claimed guarantee. 
% \jz{to delete, we don't need 'em}
% {\color{lightgray} 
\begin{restatable}{lemma}{lemLEydelstarCloseToystar}
\label{lem:y-delstar-Lip} Consider  \cref{prob:lin-eq} under \cref{assumption:linEq_smoothness} and \cref{assumption:eq}.
% \pswt{and TBD assumption}. 
Let  $\ystardel$ be as defined in \cref{eq:lower_perturb}. 
% Suppose the constraint set $Y(x):=\{y:h(x,y)=0\}$
% is convex in $x$.
% and the functions $g$ and $f$ satisfy, respectively, $\mu_{g}$-strong convexity and 
% $L_{f}$-Lipschitzness with respect to $y$, as per \cref{assumption:smoothness}. 
Then, for any $\delta\in[0,\Delta]$ with $\Delta\leq\mu_{g}/2C_{f}$, the following
relation is valid: \[
\|y_{\delta}^{*}-y^{*}\|\leq M(x)\delta,   \textrm{ with } M(x):=\frac{2}{\mu_{g}}\|\grad_{y} f(x,\ystar)\|\leq \frac{2L_f}{\mu_g}.
\]
\end{restatable}
\begin{proof}
    The first-order optimality condition applied to $g(x,y)+\delta f(x,y)$ at $\ystar$ and $\ydelstar$ gives 
\[ \innerprod{\grad_y g(x,\ydelstar)+\delta\grad_y f(x,\ydelstar)}{y^{*}-y_{\del}^{*}} \geq0,\] which upon adding and subtracting $\grad_y f(x,y^{*})$ transforms into
\[ \innerprod{\grad_y g(x,\ydelstar)+\delta[\grad_y f(x,\ydelstar)-\grad_y f(x,y^{*})]+\delta\grad_y f(x,y^{*})}{y^{*}-y_{\del}^{*}}\geq0.\numberthis\label[ineq]{eq:add-sub-fo-gpf}\] Similarly, the first-order optimality condition applied to $g$ at $\ystar$ and $y_{\delta}^{*}$ gives 
\[\innerprod{\grad_y g(x,y^{*})}{y_{\delta}^{*}-\ystar} \geq0.\numberthis\label[ineq]{eq:fo-g}\]
Adding \cref{eq:add-sub-fo-gpf} and \cref{eq:fo-g} and rearranging yields 
\begin{align*}
\innerprod{\grad_y g(x,y_{\delta}^{*})-\grad_y g(x,y^{*})+\delta[\grad_y f(x,\ydelstar)-\grad_y f(x,y^{*})]}{y_{\delta}^{*}-\ystar} & \leq\innerprod{\delta\grad_y f(x,\ystar)}{y^{*}-\ydelstar}.
\end{align*} Applying to the left side above a lower bound via strong convexity of $g+\delta f$ and to the right hand side an upper bound via Cauchy-Schwarz inequality, we have
\[s\|\ydelstar-\ystar\|\leq \delta \|\nabla_y f(x, y^*)\|, \numberthis\label[ineq]{eq:combined-fo-g-gpf}\] where $s$ is the strong convexity of $g+\delta f$. Since $f$ is $C_f$-smooth, the worst case
value of this is $s=\mu_{g}-\delta C_{f}=\mu_{g}-\frac{\mu_{g}}{2C_{f}}C_{f}=\mu_{g}/2$, which when plugged in \cref{eq:combined-fo-g-gpf} then gives the claimed bound. 
\end{proof}




\begin{restatable}{lemma}{lemLinEqLimitFiniteDiffEqualsGradF}\label{lem:lineq-in-limit-finitediff-equals-gradf}
Consider  \cref{prob:lin-eq} under \cref{assumption:linEq_smoothness} and \cref{assumption:eq}.
% \pswt{and TBD assumption}.
Then the
following relation is valid.
% \pswt{transposes in proof}
\[
\lim_{\delta\rightarrow0}\frac{\nabla_{x}[g(x,\ystardel(x))+\lamdeltar h(x,\ystar)]-\nabla_{x}[g(x,y^{*}(x))+\lamstar h(x,y^{*})]}{\delta}=\left(\frac{d y^{*}(x)}{d x}\right)^\top\nabla_{y}f(x,y^{*}(x)).
\]
\end{restatable}
\begin{proof}
Recall that by definition, $g$ is strongly convex and $y^* = \arg\min_{y: h(x,y)=0} g(x,y)$. Hence, we can apply \cref{lem:dystarDxLinEq}. Combining this with \cref{lem:non-singular-req} and further applying that linearity of $h$ implies $\nabla^2_{yy}h = 0$ and $\nabla^2_{xy}h=0$, we obtain the following: 
% Since the lower-level problem is strongly convex,  The KKT system of equations given by the first-order optimality at  $y^{*}$ is 
% \begin{align*}
% g_{y}(x,y^{*})+\lamstar h_{y}(x,y^{*}) & =0\\
% h(x,\ystar) & =0.
% \end{align*}
% Taking derivatives with respect to $x$ on both sides throughout gives us the following linear system associated with the KKT condition 
% to obtain 
% (where we used  $h_{yy}=0$ and $h_{xy}=0$ by linearity of $h$):
% \[
% \underbrace{\begin{bmatrix}g_{yy}(x,y^{*}) & h_{y}(x,y^{*})^{\top}\\
% h_{y}(x,y^{*}) & 0
% \end{bmatrix}}_{H_x(x,y^{*})}\begin{bmatrix}\frac{d\ystar}{dx}\\
% \frac{d\lamstar}{dx}
% \end{bmatrix}=\begin{bmatrix}-g_{yx}(x,y^{*})\\
% -h_{x}(x,\ystar)
% \end{bmatrix}. \numberthis\label{eq:linEq-Hxxystar-dydx}
% \]
% Rearranging the terms above and using the existence  of  $[H_x(x,y^{*})]^{-1}$  from  
\[ \begin{bmatrix}\frac{d\ystar}{dx}\\
\frac{d\lamstar}{dx}
\end{bmatrix}=\begin{bmatrix}\nabla^2_{yy} g(x,y^{*}) & \nabla_y h(x,y^{*})^{\top}\\
\nabla_y h(x,y^{*}) & 0
\end{bmatrix}^{-1}\begin{bmatrix}-\nabla^2_{yx}g(x,y^{*})\\
-\nabla_x h(x,\ystar)
\end{bmatrix}.\]
% \pswt{this proof until here repeats \cref{sec:warmup-lin-eq}; keep only one} 
So we can express the right-hand side of the claimed equation in the lemma statement by 
\begin{align*}
\left(\frac{d y^{*}(x)}{d x}\right)^\top\nabla_{y}f(x,y^{*}(x))&=\begin{bmatrix} \left(\frac{dy^*}{dx}\right)^\top & \left(\frac{d\lamstar}{dx}\right)^\top\end{bmatrix}\begin{bmatrix}\nabla_{y}f(x,y^{*}(x))\\ 
0\end{bmatrix},
\end{align*} which can be further simplified to \[\begin{bmatrix}-\nabla^2_{yx} g(x,y^{*})^\top & -\nabla_x h(x,\ystar)^\top\end{bmatrix}\begin{bmatrix}\nabla^2_{yy} g(x,y^{*}) & \nabla_y h(x,y^{*})^{\top}\\
\nabla_y h(x,y^{*}) & 0
\end{bmatrix}^{-1}\begin{bmatrix}\nabla_{y}f(x,y^{*}(x))\\ 
0\end{bmatrix}.\numberthis\label{prop3eq:RHS}\]
We now apply \cref{lem:dystarDxLinEq} to the perturbed problem defined in \cref{eq:lower_perturb}.
We know from \cref{lem:y-delstar-Lip} that $\lim_{\delta\rightarrow0}\ydelstar=\ystar$. 
The associated KKT system is given by
% \begin{equation}
\begin{align*}
\delta \nabla_y f (x,\ydelstar)+\nabla_y g(x,\ydelstar)+\nabla_y \langle\lamdeltar,  h(x,\ydelstar)\rangle  =0 \text{ and }
h(x,\ydelstar) =0. \numberthis\label{eqs:lineq-kkt-perturbed}
\end{align*}
% \end{equation}
Taking the derivative with respect of \cref{eqs:lineq-kkt-perturbed} gives the following  implicit  system, where we used the fact that $h$ is linear and hence $\nabla^2_{yy} h=0$:  
\begin{equation}
\underbrace{\begin{bmatrix}\delta \nabla^2_{yy} f(x,\ydelstar)+\nabla^2_{yy} g(x,\ydelstar) & \nabla_y h(x,\ydelstar)^{\top}\\
\nabla_y h(x,\ydelstar) & 0
\end{bmatrix}}_{H_{\delta}}\begin{bmatrix}\frac{d\ydelstar}{d\delta}\\
\frac{d\lamdeltar}{d\delta}
\end{bmatrix}=\begin{bmatrix}-\nabla_y f(x,\ydelstar)^\top\\
0
\end{bmatrix}.\label{eq:implicit_function}
\end{equation}
For a sufficiently small  $\delta$, we have $\nabla^2_{yy} g(x,\ystardel)+\delta  \nabla^2_{yy} f(x,\ydelstar)\succeq \tfrac{\mu_{g}}{2}I$, which implies  invertibility of $H_{\delta}$ by an application of \cref{{lem:non-singular-req}}. 
% $H_{\delta}\succeq\begin{bmatrix}\half\mu_{yy}I & A^{\top}\\
% A & 0
% \end{bmatrix}$, i.e., $H_{\delta}$ is positive definite. 
Since \cref{lem:y-delstar-Lip} implies  $\lim_{\delta\rightarrow0}\ydelstar=\ystar$, we get 
\[
\begin{bmatrix}\frac{d\ydelstar}{d\delta}\\
\frac{d\lamdeltar}{d\delta}
\end{bmatrix}|_{\delta=0}=\begin{bmatrix}\nabla^2_{yy} g(x,\ystar) & \nabla_y h(x,\ystar)^{\top}\\
\nabla_y h(x,\ystar) & 0
\end{bmatrix}^{-1}\begin{bmatrix}-\nabla_y f(x,\ystar)\\
0
\end{bmatrix}.
\]
So we can express the left-hand side of the expression in the lemma statement by 
\begin{align*}
&\lim_{\delta\rightarrow0}\frac{\nabla_{x}[g(x,\ystardel(x))+\langle\lamdeltar, h(x,y^*)\rangle]-\nabla_{x}[g(x,y^{*}(x))+\langle\lamstar, h(x,y^{*})\rangle]}{\delta}\\
&= \nabla^2_{xy} g(x,y^*) \frac{d\ydelstar}{d\delta} + \nabla_x h(x,y^*)^\top \frac{d\lamdeltar}{d\delta}\\ 
&=\begin{bmatrix}\nabla^2_{xy} g(x,\ystar) & \nabla_x h(x,\ystar)^\top\end{bmatrix}\begin{bmatrix}\nabla^2_{yy} g(x,\ystar) & \nabla_y h(x,\ystar)^{\top}\\
\nabla_y h(x,\ystar) & 0
\end{bmatrix}^{-1}\begin{bmatrix}-\nabla_y f(x,\ystar)\\
0
\end{bmatrix},
\end{align*}
which matches \cref{prop3eq:RHS} (since $(\nabla^2_{yx} g)^\top=\nabla^2_{xy}g$), thus concluding the proof.
\end{proof}

% }

\lemYdelstarLamdelstarSmooth*
\begin{proof} 
Rearranging \cref{{{eq:def_matrix_H}}} and applying \cref{cor:nonsingularH}, we have \[
\begin{bmatrix}\frac{d\ystar}{dx}\\
\frac{d\lamstar}{dx}
\end{bmatrix}=\begin{bmatrix}\nabla^2_{yy} g(x,y^{*}) & B^{\top}\\
B & 0
\end{bmatrix}^{-1}\begin{bmatrix}-\nabla^2_{yx} g(x,y^{*})\\
-\nabla_x h(x,\ystar)
\end{bmatrix}. 
\] This implies a Lipschitz bound of $C_H \cdot (\gssmooth + \|A\|)$. 
    Next, note that in the case with linear equality constraints, the terms in  \cref{{{eqs:differentiated-kkt-lin-eq}}}  involving second-order derivatives of $h$ are all zero; differentiating \cref{{{eqs:differentiated-kkt-lin-eq}}}    with respect to $x$, we notice that the linear system we get again has the same matrix $H$ from before. We can therefore again perform the same inversion and apply the bound on $\|H^{-1}\|$ and on the third-order derivatives of $g$ (\cref{assumption:eq})
    %\pswt{NOTE we are hiding away all the third order ugliness here}
%     we get 
%     \begin{align*}
%         g_{xyy} \frac{dy^*}{dx} + \frac{dy^*}{dx}^\top g_{yyy} \frac{dy^*}{dx} + \nabla^2_{yy} g \frac{d^2y^*}{dx^2} + h_y \frac{d^2\lamstar}{dx^2} &= -g_{xyx} - g_{yyx} \frac{dy^*}{dx} \\
%         h_y \frac{d^2y^*}{dx^2} &= 0.
%     \end{align*} Rearranging the above system of equations gives, for $H$ as in \cref{eq:def_matrix_H}, 
%     \[
% \begin{bmatrix}\frac{d^2\ystar}{dx^2}\\
% \frac{d^2\lamstar}{dx^2}
% \end{bmatrix}=H_{}^{-1}\begin{bmatrix}-g_{xyx}(x,y^{*}) - g_{yyx}\frac{dy^*}{dx} - g_{xyy}\frac{dy^*}{dx} -  \frac{dy^*}{dx}^\top g_{yyy} \frac{dy^*}{dx} \\ 
% 0 
% \end{bmatrix}.
% \] By applying TBD 
% \pswt{put the equality-specific assumption} 
% /that bounds the third-order derivatives of $g$, 
to observe that $\|\frac{d^2y^*}{dx^2}\|\leq O(C_H \cdot \gtsmooth \|\frac{dy^*}{dx}\|^2)= O(C_H^3\cdot\gtsmooth\cdot(\gssmooth+\|A\|)^2)$, where we are hiding numerical constants in the Big-Oh notation. 

As a result, we can calculate the Lipschitz smoothness constant associated with the hyper-objective $F$ by
\begin{align*}
&\|\nabla F(x) - \nabla F(\bar x)\| \\
&\leq \|\frac{dy^*(x)}{dx}\nabla_y f(x,y^*(x))-\frac{dy^*(\bar x)}{dx}\nabla_y f(\bar x,y^*(\bar x))\|+ \|\nabla_x f(x,y^*(x)) - \nabla_x f(\bar x, y^*(\bar x))\|\\
&\leq  [C_fC_H (L_g + \|A\|) + C_f C^2_H (L_g + \|A\|)^2+ L_f C_H^3 S_g (L_g+\|A\|)^2]\|x-\bar x\| \\
&\ \ +[C_f + C_f C_H (L_g +\|A\|)] \|x-\bar x\|\\
&\leq \underbrace{ 2(L_f +C_f+C_g)C_H^3 S_g (L_g +\|A\|)^2}_{C_F}\|x-\bar x\|.
\end{align*}

\end{proof}


%\jz{To be fixed, what's the exact constant dependence? We need it to set the oracle accuracy.}
\lemLineqFiniteDiffEqualsGradF*
\begin{proof}
For simplicity, we adopt the following notation throughout this proof:
$g_{xy}(x,y) = \grad^2_{xy} g,$ and $g_{xyy}$ denotes the tensor such that its $ijk$ entry is given by $\frac{\partial^3 g}{\partial x_i \partial y_j \partial y_k}$. 
We first consider the terms involving $g$. 
By the fundamental theorem of calculus, we have  \[ \nabla_{x}g(x,\ystardel(x))-\nabla_{x}g(x,y^{*}(x)) =\int_{t=0}^{\delta}g_{xy}(x,y_{t}^{*}(x))\frac{dy_{t}^{*}(x)}{dt} dt. \] As a result, we have 
\begin{align*}
&\frac{\nabla_{x}g(x,\ystardel(x))-\nabla_{x}g(x,y^{*}(x))}{\delta}-g_{xy}(x,y^{*}(x))\frac{dy_{t}^{*}(x)}{dt}|_{t=0}\\
& =\frac{1}{\delta}\int_{t=0}^{\delta}\left(g_{xy}(x,y_{t}^{*}(x))\frac{dy_{t}^{*}(x)}{dt}-g_{xy}(x,y^{*}(x))\frac{dy_{t}^{*}(x)}{dt}|_{t=0} \right)dt\\
 & =\frac{1}{\delta}\int_{t=0}^{\delta}\left(g_{xy}(x,y_{t}^{*}(x))\frac{dy_{t}^{*}(x)}{dt}-g_{xy}(x,y^{*}(x))\frac{dy_t^{*}(x)}{dt}|_{t=0}\right) dt\\
 & =\frac{1}{\delta}\int_{t=0}^{\delta}\left(g_{xy}(x,y_{t}^{*}(x))-g_{xy}(x,y^{*}(x))\right) \frac{dy_{t}^{*}(x)}{dt} dt +\frac{1}{\delta}\int_{t=0}^{\delta}g_{xy}(x,y^{*}(x))\cdot\left(\frac{dy_{t}^{*}(x)}{dt}-\frac{dy_t^{*}(x)} {dt}|_{t=0}\right)dt.\numberthis\label{eq:LE-findiff-to-ftimespartial} \end{align*} We now bound each of the terms on the right-hand side of \cref{eq:LE-findiff-to-ftimespartial}. 
 For the first term, we have  
 \begin{align*}
&\|\frac{1}{\delta}\int_{t=0}^{\delta}\left(g_{xy}(x,y_{t}^{*}(x))-g_{xy}(x,y^{*}(x))dt\right) \frac{dy_{t}^{*}(x)}{dt}\|\\
&\leq\frac{1}{\delta}\int_{t=0}^{\delta}\|\frac{dy_{t}^{*}(x)}{dt}\|\cdot\int_{s=0}^{t}\|g_{xyy}(x,y_{s}^{*}(x))\|\|\frac{dy_{s}^{*}(x)}{ds}\|ds\cdot dt\\
&\leq\frac{1}{\delta}\int_{t=0}^{\delta}\|\frac{dy_{t}^{*}(x)}{dt}\|\cdot\max_{s\in[0,\delta]}\|g_{xyy}(x,y_{s}^{*}(x))\|\cdot\|\frac{dy_{s}^{*}(x)}{ds}\|tdt\\
&\leq\frac{1}{\delta}\cdot\max_{u\in[0,\delta]}\|g_{xyy}(x,y_{u}^{*}(x))\|\cdot\delta^{2}\cdot\max_{t\in[0,\delta]}\|\frac{dy_{t}^{*}(x)}{dt}\|^{2}\\&\leq\del\cdot\max_{u\in[0,\delta]}\|g_{xyy}(x,y_{u}^{*}(x))\|\cdot\max_{t\in[0,\delta]}\|\frac{dy_{t}^{*}(x)}{dt}\|^{2}
 \\
 &= \del\cdot\gtsmooth\cdot \ystarliplineq^2,\numberthis\label[ineq]{eq:finite-diff-grad-first_term}
 \end{align*} where $\ystarliplineq$ is the Lipschitz bound on $y^*$ as shown in \cref{lem:smoothness_of_ydelstar_lamdelstar}, and $\gtsmooth$ is the smoothness of $g$ from
 \cref{assumption:eq}.
 % \pswt{cref the third order assumption}. 
 For the second term on the right-hand side of \cref{eq:LE-findiff-to-ftimespartial}, we have  
 \begin{align*}
\|\frac{1}{\delta}\int_{t=0}^{\delta}g_{xy}(x,y^{*}(x))\cdot\left(\frac{dy_{t}^{*}(x)}{dt}-\frac{dy^{*}(x)}{dt}\right)\| & \leq\frac{1}{\delta}\cdot\|g_{xy}(x,y^{*}(x))\|\cdot\int_{t=0}^{\del}\left(\int_{s=0}^{t}\|\frac{d^{2}}{ds^{2}}y_{s}^{*}(x)\|ds\right)dt \\
 & \leq\frac{1}{\del}\cdot\|g_{xy}(x,y^{*}(x))\|\cdot\max_{s\in[0,\delta]}\|\frac{d^{2}}{ds^{2}}y_{s}^{*}(x)\|\cdot\delta^{2} \\
 & \leq\delta\cdot\|g_{xy}(x,y^{*}(x))\|\cdot\max_{s\in[0,\delta]}\|\frac{d^{2}}{ds^{2}}y_{s}^{*}(x)\|\\
 &= \delta\cdot \gssmooth\cdot \ystarsmoothlineq, \numberthis\label[ineq]{eq:finite-diff-grad-second_term} 
\end{align*} where $\gssmooth$ is the bound on smoothness of $g$ as in
\cref{assumption:eq},
% \pswt{cref the smoothness assumption}, 
and $\ystarsmoothlineq$ is the bound on $\|\frac{d^2y^*}{dx^2}\|$ from \cref{lem:smoothness_of_ydelstar_lamdelstar}. 
For the terms involving the function $h$, we have \begin{align*}
\|\frac{\lamdeltar-\lam^{*}}{\delta}-\frac{d\lam_{\delta}^{*}}{d\del}|_{\delta=0}\| & =\frac{1}{\delta}\int_{t=0}^{\del}\|\frac{d\lam_{t}^{*}}{dt}-\frac{d\lam_{\delta}^{*}}{d\del}|_{\delta=0}\|dt\\
 & =\frac{1}{\delta}\int_{t=0}^{\delta}\int_{s=0}^{t}\|\frac{d^{2}}{ds^{2}}\lam_{s}^{*} \| ds\cdot dt\\
 & \leq\frac{1}{\delta}\max_{s\in[0,\delta]}\|\frac{d^{2}}{ds^{2}}\lam_{s}^{*}\|\cdot\delta^{2}\leq\delta\cdot\max_{s\in[0,\delta]}\|\frac{d^{2}}{ds^{2}}\lam_{s}^{*}\|\\
 &= \delta\cdot \lamstarsmoothlineq,\numberthis\label[ineq]{eq:finite-diff-grad-third-term}
\end{align*} where $\lamstarsmoothlineq$ is the bound on $\|\frac{d^2 \lam^*}{ds^2}\|$ from \cref{lem:smoothness_of_ydelstar_lamdelstar}. 
Combining  \cref{eq:LE-findiff-to-ftimespartial}, \cref{eq:finite-diff-grad-first_term}, \cref{eq:finite-diff-grad-second_term}, and \cref{eq:finite-diff-grad-third-term}, along with \cref{{lem:lineq-in-limit-finitediff-equals-gradf}}, \cref{cor:nonsingularH}, and  \cref{{lem:smoothness_of_ydelstar_lamdelstar}}, we have that overall bound is \begin{align*}\delta\cdot( \gtsmooth \ystarliplineq^2 + \gssmooth \ystarsmoothlineq + \lamstarsmoothlineq) &\leq O(\delta\cdot(\gtsmooth \cdot C_H^3 \cdot (\gssmooth + \|A\|)^2\cdot(\gssmooth+C_f + L_f))).\end{align*} 
% which concludes the proof.
% \pswt{Explicit scaling factors of $\delta$}
\end{proof}

% \jz{Another small lemma to show $\hat{\mathcal{G}}_{y^*}(x;\delta):=\frac{\hat{y}^*[x+\delta \nabla_y f(x,\hat{y}^*(x))]-\hat{y}^*(x)}{\delta}$ approximates $\mathcal{G}_{y^*}(x;\delta):=\frac{y^*[x+\delta \nabla_y f(x,y^*(x))]-y^*(x)}{\delta}$, that also decides the accuracy we have to get for the approximate $\hat{y}^*$.}
\subsection{Cost of linear equality constrained bilevel program}\label{sec:LEQ-main-thm-full-proof}

% Smoothness Constant
\begin{algorithm}[h]\caption{The Fully First-Order Method for Bilevel Equality Constrained Problem}\label{alg:LE-full-alg}
\begin{algorithmic}[1]
% \State \jz{state the accuracy we need to solve for the sub-problems}
\State \textbf{Input:}
Current $x_0$, accuracy $\epsilon$, perturbation $\delta = \epsilon^2/8C^2_F R_\mathcal{X}$ with $C_F= 2(L_f +C_f+C_g)C_H^3 S_g (L_g +\|A\|)^2$,  accuracy for the lower level problem $\tilde\delta = 2(C_g +\|A\|)\delta^2$.
\For{t=0,1,2,...}

\State Run \cref{alg:LE-approximate-prima-dual-solution}   to generate $\tilde \delta$-accurate primal and dual solutions $(\hat{y}^*, \hat{\lambda}^*)$ for $$\min_{y: Ax_t+By=b} g(x_t,y)$$
\State  Run \cref{alg:LE-approximate-prima-dual-solution} to generate $\tilde \delta$-accurate primal and dual solutions $(\hat{y}_\delta^*, \hat{\lambda}_\delta^*)$ for 
$$\min_{y: Ax_t+By=b} g(x_t,y)+\delta f(x_t, y)$$ 

\State Compute $\hat{v}_t:= \frac{\nabla_{x}[g(x_t,\hat{y}_\delta^*)+\hat{\lambda}_\delta^* h(x,\hat{y}^*)]-\nabla_{x}[g(x_t,\hat{y}^*)+\hat{\lambda}^* h(x,\hat{y}^*)]}{\delta}$, set $$\widetilde{\nabla} F(x_t) := \hat{v}^t + \nabla_x f(x, \hat{y}^*(x)).$$
% \pswt{cref a standalone expression}
\State Set $x_{t+1} \leftarrow \arg\min_{z\in \mathcal{X}} \| z- (x_{t}-\frac{1}{C_F}\widetilde{\nabla} F(x_{t}))\|^2.$
\EndFor

\end{algorithmic}
\end{algorithm}


\linEqFullCost*
\begin{proof}

We first show the inexact gradient $\widetilde{\nabla}F(x_t)$ generated in \cref{alg:LE-full-alg} is an $\delta$-accurate approximation to the hyper-gradient $\nabla F(x_t).$ Consider the inexact gradient defined in \eqref{eq:part-hypergrad-approx-lin-eq}
\begin{align*}
    \|v_{t}-\hat{v}_{t}\|&\leq\frac{1}{\delta}\{\|[\nabla_{x}g(x_{t},\hat{y}_{\delta}^{*})-\nabla_{x}[g(x_{t},\hat{y}^{*})]-[\nabla_{x}g(x_{t},y_{\delta}^{*})-\nabla_{x}[g(x_{t},y^{*})\|\\&\,\,\,+\ensuremath{\|\hat{\lambda}_{\delta}^{*}-\hat{\lambda}^{*}-[\lambda_{\delta}^{*}-\lambda^{*}\|\|A\|\}}\\&\leq\frac{2}{\delta}[C_{g}+\| A\|]\tilde{\delta}.
\end{align*}
Thus we get 
\begin{align*}
    \|\widetilde{\grad}F(x_{t})-\nabla F(x_{t})\|&\leq\|\grad_{x}f(x_{t},y^{*})-\grad_{x}f(x_{t},\hat{y}^{*})\|+\norm{\hat{v}^{t}-v^{t}}+\|v^{t}-\frac{dy^{*}(x^{t})}{dx}\grad_{y}f(x_{t},y^{*}(x_{t}))\|\\&\leq C_{f}\tilde{\delta}+\frac{2}{\delta}[C_{g}+\norm A]\tilde{\delta}+C_{F}\delta\\&\leq\frac{2\tilde{\delta}}{\delta}[C_{f}+C_{g}+\|A\|]+C_{F}\delta\\&\leq\frac{\epsilon^{2}}{4C_{F}R_{\mathcal{X}}}.
\end{align*}

Applied to the $C_F$-smooth hyper-objective $F$, such an inexact gradient oracle satisfies the requirement for  \cref{pr:inexact-pgd}. Thus an $\epsilon$-stationary point with $\|\mathcal{G}_F(x^t)\|\leq \epsilon$ (see Eq. \eqref{eq:gradient-mapping}  for the definition of gradient mapping) must be found in  $N=O(\frac{C_F (F(x^0)-F^*)}{\epsilon^2})$ iterations. Noting the evaluation of inexact solutions $(\hat y^*, \hat \lambda^*, \hat y_\delta^*, \hat \lambda_\delta^* )$ requires $\tilde{O}(\sqrt{C_g/\mu_g})$ first order oracle evaluations, we arrive at the total oracle complexity of $\tilde{O}(\sqrt{C_g/\mu_g}\frac{C_F (F(x^0)-F^*)}{\epsilon^2})$ for finding an $\epsilon$-stationary point. 

  %  We run projected gradient descent with the inexact gradient oracle computed via \cref{{alg:LE-inexact-gradient-oracle}}. The analysis is standard, but we provide it below for completeness. First, recall that by \cref{{lem:smoothness_of_ydelstar_lamdelstar}}, we have that $F$ is a $\beta$-smooth function, where $\beta$ is a function of $C_g$, $C_f$, and $C_H$. The iterates of our algorithm are $\xkp=\xk-\pq(\xk-\frac{1}{\beta}\widetilde{\nabla}f(\xk)),$ where $\|\nabla F(x)-\widetilde{\nabla} F(x)\|\leq \delta$. 
%Define the gradient mapping $\widetilde{G}_{\beta}(\xk)=\beta(\xk-\pq(\xk-\frac{1}{\beta}\widetilde{\nabla}f(\xk))).$
%Then by $\beta$-smoothness of $F$, we have 
% \begin{align*}
% f(\xkp)  =f(\xk-\frac{1}{\beta}\widetilde{G}_{\beta}(\xk))
%  & \leq f(\xk)-\frac{1}{\beta}\widetilde{G}_{\beta}(\xk)^{\top}\nabla f(\xk)+\frac{1}{2\beta}\|\widetilde{G}_{t}(\xk)\|^{2}\\
%  & =f(\xk)-\frac{1}{2\beta}\|\widetilde{G}_{\beta}(\xk)\|^{2}+\frac{1}{\beta}\widetilde{G}_{\beta}(\xk)^{\top}(\widetilde{G}_{\beta}(\xk)-\nabla f(\xk)).\numberthis\label[ineq]{eq:fn_decrease_lineq_pgd}
% \end{align*}
% We now show that $\frac{1}{\beta}\widetilde{G}_{\beta}(\xk)^{\top}(\widetilde{G}_{\beta}(\xk)-\nabla f(\xk))\leq0.$ Let $\widetilde{y}_k = \xk-\frac{1}{\beta}\widetilde{\nabla}f(\xk)$, and let $y_k = \xk-\frac{1}{\beta}{\nabla}f(\xk)$. 
% Then have that 
% \begin{align*}
% \frac{1}{\beta}\widetilde{G}_{\beta}(\xk)^{\top}(\widetilde{G}_{\beta}(\xk)-\nabla f(\xk)) & =\beta(\xk-\pq(\widetilde{y}_k))^{\top}(y_k - \pq(\widetilde{y}_k))\\
%  & = \beta(\xk-\pq(\widetilde{y}_k))^\top (\widetilde{y}_k - \pq(\widetilde{y}_k)) \\
%  &\quad +\beta(\xk-\pq(\widetilde{y}_k))^\top(y_k - \widetilde{y}_k)\\
% &\leq \beta(\xk-\pq(\widetilde{y}_k))^\top(y_k - \widetilde{y}_k)\\
% &\leq \delta \beta R,
% \end{align*} where the penultimate inequality uses the fact that $\mathcal{X}$ is a convex set, and $R$ is the diameter of the set $X$. Combining this with \cref{eq:fn_decrease_lineq_pgd}, we have that the function decrease per iteration is \[ f(\xkp)\leq f(\xk)-\frac{1}{2\beta}\|\widetilde{G}_{\beta}(\xk)\|^{2} + \delta \beta R. \] Summing over $T=\widetilde{O}(\epsilon^{-2})$ iterations telescopes the terms and implies the desired stationarity. 
\end{proof}
\subsection{The cost of inexact projected gradient descent method}
In this subsection, we state the number of iterations required by  projected gradient descent method to find an $\epsilon$-stationary point using inexact gradient oracles. Specifically, we consider the following non-convex smooth problem  where the objective $F$ is assumed to be $C_F$-Lipschitz smooth: 
\begin{equation}\label{eq:prob-smooth-constrained}
    \mbox{minimize}_{x\in \mathcal{X}} F(x).
\end{equation}
Since the feasible region $\mathcal{X}$ is compact, we use the norm of the following gradient mapping $\mathcal{G}_F(x)$ as the stationarity criterion 
\begin{equation}\label{eq:gradient-mapping}
    \mathcal{G}_F(x):= {C_F}(x - x^+) \text{ where } x^+ = \arg\min_{z\in \mathcal{X}} \left\| z-\left(x -\frac{1}{C_F} \nabla F(x)\right)\right\|^2.
\end{equation}
Initialized to some $x_0$ and the inexact gradient oracle $\widetilde{\nabla} F$, the updates of the inexact projected gradient descent method is given by 
\begin{equation}\label{alg:inexact-projected-gd-algorithm}
    \begin{split}
        \text{\textbf{For}}&\text{ t=1,2,..., N \textbf{do}:}\\
        &\text{Set } x_t \leftarrow \arg\min_{z\in \mathcal{X}} \left\| z- \left(x_{t-1}-\frac{1}{C_F}\widetilde{\nabla} F(x_{t-1})\right)\right\|^2.
    \end{split}
\end{equation}
The next proposition calculates the complexity result. 
\begin{proposition}\label{pr:inexact-pgd}
    Consider the constrained optimization problem in \eqref{eq:prob-smooth-constrained} with $F$ being $C_F$-Lipschitz smooth and $\mathcal{X}$ having a radius of $R$. When supplied with a $\delta=\epsilon^2/4C_F R$ -inexact gradient oracle $\widetilde{\nabla} F$, that is, $\|\nabla F(x)-\widetilde{\nabla}F(x)\|\leq \delta$, the solution generated by the projected gradient descent method \eqref{alg:inexact-projected-gd-algorithm} satisfies 
    $$\min_{t \in [N]} \| \mathcal{G}_F(x_t)\|^2 \leq \frac{C_F(F(x_0)-F^*)}{N}+\delta C_F R,$$
    that is, it takes at most $O(\frac{C_F (F(x^0)-F^*)}{\epsilon^2})$ iterations to generate some $\bar x$ with $\|\mathcal{G}_F(x)\|\leq \epsilon$.

\end{proposition}
\begin{proof}
    By $C_F$-smoothness of $F$, we have 
\begin{align*}
f(x_{t+1})  =f(x_t-\frac{1}{C_F}\widetilde{\mathcal{G}_F}(x_t))
 & \leq f(x_t)-\frac{1}{C_F}\widetilde{\mathcal{G}_F}(x_t)^{\top}\nabla f(x_t)+\frac{1}{2C_F}\|\widetilde{\mathcal{G}_F}(x_t)\|^{2}\\
 & =f(x_t)-\frac{1}{2C_F}\|\widetilde{\mathcal{G}_F}(x_t)(x_t)\|^{2}+\frac{1}{C_F}\widetilde{\mathcal{G}_F}(x_t)^{\top}(\widetilde{\mathcal{G}_F}(x_t)-\nabla f(x_t)).\numberthis\label[ineq]{eq:fn_decrease_lineq_pgd}
\end{align*}
We now show that $\frac{1}{\beta}\widetilde{\mathcal{G}_F}(x_t)^{\top}(\widetilde{\mathcal{G}_F}(x_t)-\nabla f(x_t))\leq0.$ Let $\widetilde{y}_t = x_t-\frac{1}{C_F}\widetilde{\nabla}F(x_t)$, and let $y_t = x_t-\frac{1}{C_F}{\nabla}f(x_t)$. 
Then have that 
\begin{align*}
\frac{1}{C_F}\widetilde{\mathcal{G}_F}(x_t)^{\top}(\frac{1}{C_F}\widetilde{\mathcal{G}_F}(x_t)-\nabla f(x_t)) & =C_F(x_t-\pq(\widetilde{y}_t))^{\top}(y_t - \pq(\widetilde{y}_t))\\
 & = C_F(x_t-\pq(\widetilde{y}_t))^\top (\widetilde{y}_t - \pq(\widetilde{y}_t)) \\
 &\quad +C_F(x_t-\pq(\widetilde{y}_t))^\top(y_t - \widetilde{y}_t)\\
&\leq C_F(x_t-\pq(\widetilde{y}_t))^\top(y_t - \widetilde{y}_t)\\
&\leq \delta C_F  R,
\end{align*}
where the penultimate inequality uses the fact that $\mathcal{X}$ is a convex set, and $R$ is the diameter of the set $X$. Combining this with \cref{eq:fn_decrease_lineq_pgd}, we have that the function decrease per iteration is 
\[ F(x_{t+1})\leq F(x_t)-\frac{1}{2C_F}\|\widetilde{\mathcal{G}_F}(x_t)\|^{2} + \delta C_F R. \] 

Summing over $N$ iterations telescopes the terms, we get 
$$\min_{t\in[N]}\|\widetilde{\mathcal{G}_F}(x_t)\|^{2}\leq \frac{1}{N} C_F (F(x^0)-F^*) + \delta C_F R.$$
Substituting in $N=\frac{4}{\epsilon^2}C_F (F(x^0)-F^*)$  and the choice of $\delta = \epsilon^2/4C_F R$, we get 
$$\min_{t\in[N]}\|\widetilde{\mathcal{G}_F}(x_t)\|^{2}\leq \frac{\epsilon^2}{2}.$$
Taking into account the fact that $\|\widetilde{\mathcal{G}_F}(x_t)- \mathcal{G}_F (x_t) \|\leq \|\nabla F(x^t)-\widetilde{\nabla} F(x^t)\|\leq \delta$,  we obtain the desired result. 

\end{proof}

\subsection{The cost of generating approximate solutions to the linearly constrained LL problem}\label{sec:LEQ-cost-computing-ystar-lamstar}
In this subsection, we address the issue of generating approximations to the primal and dual solutions  $(y^*,\lambda^*)$ associated with the lower-level problem in \cref{{{prob:lin-eq}}}. These approximations are required for computing the approximate hypergradient in \cref{alg:LE-inexact-gradient-oracle}.  For notational simplicity, we are going to consider the following constrained strongly convex problem: 
\[ 
\begin{array}{ll}
    \mbox{minimize}_{y\in\R^d} &g(y)\\
    \mbox{subject to } & By=b.
\end{array}\numberthis\label[prob]{eq:simple-linear-cosntrained-problem}
\] 

We propose the following simple scheme to generate approximate solutions to \cref{{eq:simple-linear-cosntrained-problem}}. 
\begin{center}
\fbox{\begin{varwidth}{\dimexpr\textwidth-2\fboxsep-2\fboxrule\relax}
  Compute a feasible $\hat{y}$  such that $\|\hat{y}-y^*\|\leq\delta$.
  Then solve
  
     \begin{equation}\label{eq:approximate-lam-hat}
        \hat\lambda = \arg\min_{\lambda \in \R^m} \|\nabla_y g(\hat y)-B^\top\lambda\|^2.
    \end{equation}
    
\end{varwidth}}
\end{center}

The following lemma tells us that $\hat \lambda$ is close to $\lambda^*$ if $B$ has full row rank. 

\begin{lemma}\label{lm:generating-lamhat}
    Suppose $g$ in \cref{eq:simple-linear-cosntrained-problem} is a $C_g$-Lipschitz smooth, and the matrix $B$ has full row rank such that the following matrix $M_{B}$ is invertible
    $$M_B =\begin{bmatrix}
        I & B^\top\\
        B & 0
    \end{bmatrix}.$$
    Then the approximate solution $(\hat \lambda, \hat y)$ from \cref{eq:approximate-lam-hat} satisfies $\|\hat\lambda -\lambda^*\|\leq \|M_{B}^{-1}\| (1+C_g)\delta$.
\end{lemma}

\begin{proof}
    Since $(\lambda^*,y^*)$ satisfy the KKT conditions, they are the solution to the following linear system 
    \begin{equation}\label{lmeq:ystarlamstar-equation}
        \underbrace{\begin{bmatrix}I & B^\top\\ B & 0 \end{bmatrix}}_{=M_B} 
    \begin{bmatrix} y^* \\ \lambda^*\end{bmatrix}
    =\begin{bmatrix}-\nabla_y g(y^*) + I y^* \\ b\end{bmatrix}.
    \end{equation}
    That is %\pswt{justify invertibility of $M_B$}\jz{we can show the only solution to $M_B [y;\lambda] =0$ is zero.}
    $$ \begin{bmatrix} y^* \\ \lambda^*\end{bmatrix}
    =M_B^{-1}\begin{bmatrix}-\nabla_y g(y^*) + I y^* \\ b\end{bmatrix}.
    $$
    On the other hand, the approximate solutions $(\hat{y}, \hat{\lambda})$ in \cref{eq:approximate-lam-hat} satisfies %\pswt{sign of $-B^\top \hat{\lambda}$ seems off? in the first equation below?}\jz{fixed}
        $$\begin{bmatrix}I & B^\top\\ B & 0 \end{bmatrix}
    \begin{bmatrix} \hat{y} \\ \hat{\lambda}\end{bmatrix}
    =\begin{bmatrix} B^\top \hat{\lambda} + I \hat{y} \\ b\end{bmatrix}.$$
    We show the right hand side (r.h.s) of the above equation to be close to the r.h.s of \cref{lmeq:ystarlamstar-equation}. Let $S:=\{B^\top \lambda: \lambda\in \R^m\}$ denote the subspace spanned by the rows of $B$. We can rewrite $B^\top \hat\lambda$ as the projection of $\nabla g(\hat y)$ onto $S$, that is, 
    \begin{align*}
        &B^\top \hat \lambda = \arg\min_{s\in S}\|\nabla_y g(\hat y) -s\|^2\\
     -\nabla_y g( y^*)=&B^\top \lambda^* = \arg\min_{s\in S}\|\nabla_y g( y^*) -s\|^2,
    \end{align*}
    where the second relation follows from the KKT conditon associated with $(\lambda^*,y^*)$. Since the projection is an non-expansive operation, we have 
    $$\|B^\top \hat\lambda -(- \nabla_y g(y^*))\|=\|B^\top \hat \lambda - B^\top \lambda^*\|\leq \|\nabla_y g(\hat y) -\nabla g(y^*)\|\leq C_g \|\hat y - y^*\|\leq C_g \delta.$$
    We can rewrite $(\hat y, \hat \lambda)$ as solutions to the following linear system with some $\|\tau\|\leq (1+C_g)\delta$, 
       $$ \begin{bmatrix} \hat{y} \\ \hat \lambda\end{bmatrix}
    =M_B^{-1}\begin{bmatrix}-\nabla_y g(y^*) + I y^* +\tau \\ b\end{bmatrix}.
    $$
    Thus we get 
     $$ \|\begin{bmatrix} \hat{y} \\ \hat \lambda\end{bmatrix}-\begin{bmatrix} y^* \\ \lambda^*\end{bmatrix}\|
    =\|M_B^{-1}\|\|\begin{bmatrix}\tau \\ 0\end{bmatrix}\leq \|M_B^{-1}\|(1+C_g)\delta.
    $$
    
    %\jz{For the inequality case, similar argument might still be fine if the KKT matrix is invertible. If not, all could be lost.}
    
\end{proof}

Now we can just use the AGD method to generate a close enough approximate solution $\hat y$ and call up the Subroutine in \cref{eq:approximate-lam-hat} to generate the approximate dual solution $\hat \lambda.$

\begin{algorithm}[h]\caption{The Projected Gradient Method to Generate Primal and Dual Solutions for a Linearly Constrained Problem}\label{alg:LE-approximate-prima-dual-solution}
\begin{algorithmic}[1]
\State \textbf{Input}: accuracy requirement $\epsilon>0$ and linearly constrained problem $\min_{y: By =b} g(y)$.
 \State Starting from $y^0=0$ and using $Y:=\{y\in\R^d: By=b\}$ as the simple feasible region. 
 \State Run the Accelerated Gradient Descent (AGD) Method (Section 3.3 in \cite{lan2020first}) for $N=\lceil 4\sqrt{C_g/\mu_g}\log\frac{\|y^*\| \|M^{-1}_B\|(C_g+1)}{\mu_g\epsilon} \rceil$ iterations.
  \State Use the $y^N$ as the approximate solution $\hat y$ to generate $\hat \lambda$ according to \cref{eq:approximate-lam-hat}.
  \State \Return $(\hat y, \hat \lambda)$
\end{algorithmic}
\end{algorithm}

\begin{proposition}
    Suppose the objective function $g$ is both $L_g$-smooth and $\mu_g$-strongly convex, and that the constraint satisfies the assumption in \cref{lm:generating-lamhat}. Fix an $\epsilon>0$, the solution  $(\hat y, \hat \lambda)$ returned by the above procedure satisfies $\|y^* - \hat y\|\leq \epsilon$ and $\|\hat \lambda - \lambda^*\|\leq \epsilon$. In another words, the cost of generating $\epsilon$-close primal and dual solutions are bounded by $O(\sqrt{\frac{C_g}{\mu_g}}\log{\frac{1}{\epsilon}}).$
\end{proposition}
\begin{proof}
With $N:=\lceil 4\sqrt{C_g/\mu_g}\log\frac{\|y^*\| \|M^{-1}_B\|(L_g+1)}{\mu_g\epsilon} \rceil$, Theorem 3.7 in \cite{lan2020first} shows that $\|y^N - \hat y\|\leq \epsilon/ \|M_{B}^{-1}\| (1+L_g)$. Then we can apply \cref{lm:generating-lamhat} to obtain the desired bound.
\end{proof}

\section{Proofs for \cref{sec:nonsmooth}} \label{sec:alg_proof}
Our algorithms are based on the Lipschitzness of $F$, which we prove below. 
\lemLipscConstrBilevel*
\begin{proof}
% \pswt{check!} \gk{Defer proof to \cref{sec:alg_proof}}
    By Lemma $2.1$ of \cite{ghadimi2018approximation}, the hypergradient of $F$ computed with respect to the variable $x$ may be expressed as $\nabla_x F(x) = \nabla_x f(x, y^\ast(x)) + \left(\frac{d y^\ast(x)}{dx}\right)^\top \cdot \nabla_y f(x, y^\ast(x))$. Since we impose Lipschitzness on $f$ and $y^*$, we can bound each of the terms of $\nabla_x F(x)$ by the claimed bound. 
    % \gk{Note that instead of citing, we already mention this in Section 2 so can just ref}. If we impose Lipschitzness assumptions on $f$, with respect to each of the coordinates, and additionally, if we can show $\|\nabla_x y^\ast(x)\|\leq L_{yx}$ for some constant, then it concludes the proof of Lipschitzness of $F$ with a Lipschitz constant $L_F \leq L_{fx} + L_{fy}\cdot L_{yx}$. Note that a bound on $\|\nabla_x y^\ast(x)\|$ was obtained by \cite{ghadimi2018approximation} in terms of second-order properties of $g$ and one in terms of constants from weaker assumptions in \cite{kwon2023fully} --- however, the latter also involves the Lagrange multiplier used therein. 
    % \pswt{How can we get for $y^\ast(x)$ a Lipschitz constant that is independent of any Lagrange multipliers? I think this is perhaps a self-contained question: what is the Lipschitz constant of the minimizer of a (strongly) convex function over the $0$-level set of a (strongly) convex function? Note: Jimmy has a solution for this.} 
\end{proof}


\subsection{Faster algorithm for low upper-level dimensions}\label{sec:zeroth-order-algs}
% Notably, \cref{alg: OIGRM} requires hypergradient estimates, which contributes to the overall complexity of the optimization scheme. On the other hand, estimating the \emph{value} of $F(x)$ at any given point $x$, can be done using $\widetilde{O}(1)$ oracle calls, as it amounts to solving a single strongly convex optimization problem. This is formally stated in the following lemma. \pswt{added this text to the main body}
In this section we analyze \cref{alg: IZO}, which as stated in \cref{{sec:nonsmooth}}, requires evaluating only the hyperobjective $F$ (as opposed to estimating the hypergradient in \cref{alg: OIGRM}).

The motivation for designing such an algorithm, is that
while evaluating $\nabla F$ up to $\alpha$ accuracy requires $O(\alpha^{-1})$ gradient evaluations, the hyperobjective value can be estimated at a linear rate:

\lemZeroOrderApprox*
\begin{proof}[Proof of \cref{lem:ZeroOrderApprox}]
    We note that it suffices to find $\tilde{y}^*$ such that $\norm{\tilde{y}^*-y^*(x)}\leq \alpha/L_f$, since setting $\tF(x):=f(x,\tilde{y}^*)$ will then satisfy $|\tF(x)-F(x)|=|f(x,\tilde{y}^*)-f(x,y^*(x))|\leq L_f\cdot \tfrac{\alpha}{L_f}=\alpha$ by Lispchitzness of $f$, as required. Noting that $y^*(x)=\arg\min_{h(x,y)\leq 0}g(x,y)$ is the solution to a constrained smooth, strongly-convex problem with condition number $C_g/\mu_g$, it is possible to approximate it up to $\alpha/L_f$ with $O(\sqrt{C_g/\mu_g}\log(L_f/\alpha))$ first-order oracle calls using the result of \citet{zhang2022solving}.
\end{proof}


Accordingly, we consider \cref{alg: IZO}, which is a zero-order variant of \cref{alg: OIGRM}, whose guarantee is summarized is the theorem below.



\begin{theorem} \label{thm: Lipschitz-min-with-inexact-zero-oracle}
Suppose $F:\reals^d\to\reals$ is $L$-Lipschitz, 
% $F(x_0)-\inf F\leq \Delta$
and that $|\widetilde{F}(\cdot)-F(\cdot)|\leq\alpha$.
Then running \cref{alg: OIGRM} with
$\rho=\min\left\{\tfrac{\delta}{2},\tfrac{F(x_0)-\inf F}{L}\right\},\nu=\delta-\rho,~D=\Theta\left(\frac{\nu\epsilon^2\rho^2}{d\rho^2 L^2+\alpha^2 d^2}\right),\eta=\Theta\left(\frac{\nu\epsilon^3\rho^4}{(d\rho^2L^2+\alpha^2d^2)^2}\right)$,
outputs a point $x^{\out}$ such that $\E[\mathrm{dist}(0,{\partial}_\delta F(x^{\out}))]\leq\epsilon+\alpha$ with
\[T=
O\left(\frac{d(F(x_0)-\inf F)}{\delta\epsilon^3}\cdot \left(L^2+\alpha^2 (\frac{d}{\delta^{2}}+\frac{dL^2}{(F(x_0)-\inf F)^2})\right)\right)
\text{ calls to } \tF(\cdot).\]
% Under the same setting as \citep[Theorem 7.2]{chen2023bilevel},
% suppose that $\mathrm{SGM}$ (as in \citep[Theorem 7.1]{chen2023bilevel}) is set so that $|\tF(\cdot)-\varphi(\cdot)|\leq \zeta$ for $\zeta=\Theta(\delta\epsilon/d)$. Then
% Algorithm~\ref{alg: IZO} outputs a point $\bx^{\out}$ such that $\E[\min\{\norm{s}:s\in\partial_\delta \varphi(\bx^{\out})\}]\leq\epsilon$ with
% \[
% T=O\left(\frac{d}{\delta\epsilon^3}\right)~.
% \]
\end{theorem}


% We now .
% to \cref{sec:alg_proof}.
Combining the result of \cref{thm: Lipschitz-min-with-inexact-zero-oracle} with the complexity of hyperobjective estimation,
as given by \cref{lem:ZeroOrderApprox},
we obtain convergence to a $(\delta,\epsilon)$-stationary point of  \cref{{prob:ineq}} with $\widetilde{O}(d_x\delta^{-1}\epsilon^{-3})$ gradient calls overall.
% \gk{remark about UL dimension being small in some applications of interest?}


% \pswt{the text before this comment is copied straight from what was previously in the main body and must therefore be appropriately edited for coherence}


\subsubsection{Proof of \cref{thm: Lipschitz-min-with-inexact-zero-oracle}}


Denoting the uniform randomized smoothing $F_\rho(x):=\E_{\norm{z}\leq1}[F(x+\rho\cdot z)]$ where the expectation, here and in what follows, is taken with respect to the uniform measure, it is well known \citep[Lemma 10]{shamir2017optimal} that
\begin{align}
\E_{\norm{w}=1}\left[\tfrac{d}{2\rho}(F(x+\rho w)-F(x-\rho w)) w\right]
&=\nabla F_\rho( x)~,
\nonumber
\\
\E_{ \norm{w}=1}\norm{\nabla F_\rho( x)-\tfrac{d}{2\rho}( F( x+\rho w)- F( x-\rho w)) w}^2
&\lesssim dL^2~.\label{eq:grad_var_bound}
\end{align}
We first show that replacing the gradient estimator with the inexact evaluations $\tF(\cdot)$ leads to a biased gradient estimator of $ F$.

\begin{lemma}\label{lem: inexact gradient}
Suppose $| F(\cdot)-\tF(\cdot)|\leq\alpha$. Denoting
\begin{align*}
 g_x&=\tfrac{d}{2\rho}( F( x+\rho w)- F( x-\rho w)) w~,
\\
\tbg_x&=\tfrac{d}{2\rho}(\tF(x+\rho w)-\tF( x-\rho w)) w~,
\end{align*}
it holds that
\[
\E_{ \norm{w}=1}\norm{ g_x-\tbg_x}\leq\frac{\alpha d}{\rho}~,
~~~~~\text{and}~~~~~
\E_{\norm{w}=1}\norm{\tbg_x}^2\lesssim \frac{\alpha^2 d^2}{\rho^2}+dL^2~.
\]
\end{lemma}

\begin{proof}
For the first bound, we have
\[
\E_{\norm{w}=1}\norm{g_x-\tbg_x}
\leq \frac{d}{2\rho}(2\alpha)\E_{\norm{w}=1}\norm{w}
=\frac{\alpha d}{\rho}~,
\]
while for the second bound
\[
\E_{\norm{w}=1}\norm{\tbg_x}^2
=\E_{\norm{w}=1}\norm{\tbg_x-g_x+g_x}^2
\leq 2\E_{\norm{w}=1}\norm{\tbg_x-g_x}^2+2\E_{\norm{w}=1}\norm{g_x}^2
\lesssim \frac{d^2}{\rho^2}\cdot\alpha^2+dL^2~,
\]
where the last step invoked \cref{eq:grad_var_bound}.
\end{proof}

% The next step would be to prove that online gradient descent minimizes the regret with respect to inexact evaluations. Recalling definitions from online learning, given a sequence of linear losses $\ell_t(\cdot)=\inner{\bg_t,\cdot}$, if the algorithm chooses $\bm{\Delta}_1,\dots,\bm{\Delta}_T$ we denote the regret with respect to $\bu$ as $\mathrm{Reg}_T(\bu):=\sum_{t=1}^{T}\inner{\bg_t,\bm{\Delta}_t-\bu}$. Consider an update rule according to \emph{inexact} online projected gradient descent: $\bDelta_{t+1}:=\mathrm{clip}_{D}(\bDelta_{t}-\eta_t\tbg_t)$.

% \begin{lemma}\label{lem: inexact OGD}
% In the setting above, suppose that $\norm{\tbg_t-\bg_t}\leq\alpha$ and $\norm{\tbg_t}^2\leq\widetilde{G}^2$ for all $t\in[T]$.
% Then for any $\norm{\bu}\leq D$ it holds that
% \[
% \mathrm{Reg}_T(\bu)\leq \frac{D^2}{\eta_T}+\widetilde{G}^2\sum_{t=1}^{T}\eta_t+ \alpha DT~.
% \]
% In particular, setting $\eta_t\equiv\frac{D}{\widetilde{G}\sqrt{T}}$ yields
% \[
% \mathrm{Reg}_T(\bu)\lesssim D\widetilde{G}\sqrt{T}+\alpha DT~.
% \]
% \end{lemma}

% \begin{proof}
% For any $t\in[T]:$
% \begin{align*}
% \norm{\bDelta_{t+1}-\bu}^2
% &=\norm{\mathrm{clip}_{D}(\bDelta_{t}-\eta_t\tbg_t)-\bu}^2
% \\&\leq \norm{\bDelta_{t}-\eta_t\tbg_t-\bu}^2
% =\norm{\bDelta_{t}-\bu}^2+\eta_t^2\norm{\tbg_t}^2-2\eta_t\inner{\bDelta_{t}-\bu,\tbg_t}~,
% \end{align*}
% thus
% \[
% \inner{\tbg_t,\bDelta_t-\bu}
% \leq \frac{\norm{\bDelta_t-\bu}^2-\norm{\bDelta_{t+1}-\bu}^2}{2\eta_t}+\frac{\eta_t}{2}\norm{\tbg_t}^2~,
% \]
% from which we get
% \begin{align*}
% \inner{\bg_t,\bDelta_t-\bu}
% &=\inner{\tbg_t,\bDelta_t-\bu}+
% \inner{\bg_t-\tbg_t,\bDelta_t-\bu}
% \\&\leq\frac{\norm{\bDelta_t-\bu}^2-\norm{\bDelta_{t+1}-\bu}^2}{2\eta_t}+\frac{\eta_t}{2}\widetilde{G}^2+\alpha D~.
% \end{align*}
% Summing over $t\in[T]$, we see that
% \begin{align*}
% \mathrm{Reg}_T(\bu)
% &\leq \sum_{t=1}^{T}\norm{\bDelta_t-\bu}^2\left(\frac{1}{\eta_t}-\frac{1}{\eta_{t-1}}\right)+\frac{\widetilde{G}^2}{2}\sum_{t=1}^{T}\eta_t+T\alpha D
% \\
% &\leq \frac{D^2}{\eta_T}+\widetilde{G}^2\sum_{t=1}^{T}\eta_t+ \alpha DT~.
% \end{align*}
% The simplification for $\eta_t\equiv\frac{D}{\widetilde{G}\sqrt{T}}$ readily follows.

% \end{proof}


We are now ready to analyze \cref{alg: IZO}.
% \begin{proof}[Proof of \cref{{thm: Lipschitz-min-with-inexact-zero-oracle}}]
We denote $\alpha'=\frac{\alpha d}{\rho},~\widetilde{G}=\sqrt{\frac{\alpha^2 d^2}{\rho^2}+dL^2}$. Since $x_t=x_{t-1}+\Delta_{t}$, we have
\begin{align*}
 F_\rho(x_t)- F_\rho(x_{t-1})
&=\int_{0}^{1}\inner{\nabla F_{\rho}(x_{t-1}+s\Delta_t),\Delta_t}ds
\\
&=\E_{s_t\sim\Unif[0,1]}\left[\nabla F_{\rho}(x_{t-1}+s_t{\Delta}_t),\Delta_t\right]
\\&=\E\left[\inner{\nabla F_{\rho}(z_{t}),\Delta_t}\right]~.
% \\&=\E\left[\inner{\tbg_{t},\bDelta_t}\right]+\E\left[\inner{\nabla F_{\delta}(\bz_{t})-\tbg_{t},\bDelta_t}\right]
% \\&\leq\E\left[\inner{\tbg_{t},\bDelta_t}\right]+\E\left[\norm{\nabla F_{\delta}(\bz_{t})-\tbg_{t}}\cdot\norm{\bDelta_t}\right]
% \\& \leq\E\left[\inner{\tbg_{t},\bDelta_t}\right]+D\alpha~.
\end{align*}
% where the last inequality follows from Lemma~\ref{lem: inexact gradient}.
By summing over $t\in[T]=[K\times M]$, we get for any fixed sequence $u_1,\dots,u_K\in\reals^d:$
\begin{align*}
    \inf  F_\rho\leq  F_\rho(x_T)
    &\leq  F_\rho(x_0)+\sum_{t=1}^{T}\E\left[\inner{\nabla F_{\rho}(z_{t}),\Delta_t}\right]
    \\&= F_\rho(x_0)+\sum_{k=1}^{K}\sum_{m=1}^{M}\E\left[\inner{\nabla F_{\rho}(z_{(k-1)M+m}),\Delta_{(k-1)M+m}-u_k}\right]\\
    &~~~~+\sum_{k=1}^{K}\sum_{m=1}^{M}\E\left[\inner{\nabla F_{\rho}(z_{(k-1)M+m}),u_k}\right]
    \\
    &\leq  F_\rho(x_0)+\sum_{k=1}^{K}\mathrm{Reg}_M(u_k)+\sum_{k=1}^{K}\sum_{m=1}^{M}\E\left[\inner{\nabla F_{\rho}(z_{(k-1)M+m}),u_k}\right]
    \\
    &\leq F_\rho(x_0)+KD\widetilde{G}\sqrt{M}+K\alpha'DM+\sum_{k=1}^{K}\sum_{m=1}^{M}\E\left[\inner{\nabla F_{\rho}(z_{(k-1)M+m}),u_k}\right]
\end{align*}
where the last inequality follows by combining \cref{lem: inexact gradient} and \cref{lem: inexact OGD}.
By setting $u_k:=-D\frac{\sum_{m=1}^{M}\nabla F_{\rho}(z_{(k-1)M+m})}{\norm{\sum_{m=1}^{M}\nabla F_{\rho}(z_{(k-1)M+m})}}$, rearranging and dividing by $DT=DKM$ we obtain
\begin{align}
\frac{1}{K}\sum_{k=1}^{K}\E\norm{\frac{1}{M}\sum_{m=1}^{M}\nabla F_{\rho}(z_{(k-1)M+m})}
&\leq \frac{ F_\rho(x_0)-\inf F_\rho}{DT}+\frac{\widetilde{G}}{\sqrt{M}}+\alpha'
\nonumber\\
&=\frac{ F_\rho(x_0)-\inf F_\rho}{K\nu}+\frac{\sqrt{\frac{\alpha^2 d^2}{\rho^2}+L^2d}}{\sqrt{M}}+\frac{\alpha d}{\rho}  
\nonumber\\
&\leq
\frac{ F_\rho(x_0)-\inf F_\rho}{K\nu}
+\frac{{\frac{\alpha d}{\rho}}}{\sqrt{M}}
+\frac{L\sqrt{d}}{\sqrt{M}}
+\frac{\alpha d}{\rho}~. \label{eq: bound eps}
\end{align}
Finally, note that for all $m\in[M]:\norm{z_{(k-1)M+m}-\overline{x}_{k}}\leq M D\leq\nu$, therefore 
$\nabla F_{\rho}(z_{(k-1)M+m})\in\partial_\nu F_\rho(\overline{x}_{k})\subset \partial_{\delta}F(\overline{x}_{k})$, where the last containment is due to \citep[Lemma 4]{kornowski2024algorithm} by using our assignment $\rho+\nu= \delta$.
Invoking the convexity of the Goldstein subdifferential, this implies that
\[
\frac{1}{M}\sum_{m=1}^{M}\nabla F_{\rho}(z_{(k-1)M+m})
\in\partial_{\delta}F(\overline{x}_{k})
~,
\]
thus it suffices to bound the first three summands in \eqref{eq: bound eps} by $\epsilon$ in order to finish the proof.
This happens as long as $
\frac{ F_\rho(x_0)-\inf F_\rho}{K\nu}\leq\frac{\epsilon}{3}$, $\frac{{\frac{\alpha d}{\rho}}}{\sqrt{M}}\leq\frac{\epsilon}{3}$, and $\frac{L\sqrt{d}}{\sqrt{M}}\leq\frac{\epsilon}{3}$, which imply $K\gtrsim \frac{ F_\rho(x_0)-\inf F_\rho}{\nu\epsilon}$, $ M\gtrsim \frac{\alpha^2 d^2}{\rho^2\epsilon^2}$, and $M\gtrsim \frac{L^2 d}{\epsilon^2}~$. 
By our assignments of $\rho$ and $\nu$, these  result in
\begin{align*}
T=KM
&=O\left(\frac{ F_\rho(x_0)-\inf F_\rho}{\nu\epsilon}\cdot \left(\frac{\alpha^2 d^2}{\rho^2\epsilon^2}+\frac{L^2 d}{\epsilon^2}\right)\right)
\\
&=O\left(\frac{(F(x_0)-\inf F)d}{\delta\epsilon^3}\cdot \left(\frac{\alpha^2 d}{\rho^2}+L^2\right)\right)
\\
&=O\left(\frac{(F(x_0)-\inf F)d}{\delta\epsilon^3}\cdot \left(\alpha^2 d\cdot\max\left\{\frac{1}{\delta^2},\frac{L^2}{(F(x_0)-\inf F)^2}\right\}+L^2\right)\right)
~,
\end{align*}
completing the proof.

% \end{proof}








\subsection{Proof of \cref{thm:Lipschitz-min-with-inexact-grad-oracle}}

% Denoting the uniform randomized smoothing $\varphi_\delta(x):=\E_{z\sim\B^d}[\varphi(x+\delta\cdot z)]$, it is well known \citep[Lemma 10]{shamir2017optimal} that
% \begin{align*}
% \E_{\bw\sim\mathbb{S}^{d-1}}\left[\tfrac{d}{2\delta}(\varphi(\bx+\delta\bw)-\varphi(\bx-\delta\bw))\bw\right]
% &=\nabla\varphi_\delta(\bx)~,
% \\
% \E_{\bw\sim\mathbb{S}^{d-1}}\norm{\nabla\varphi_\delta(\bx)-\tfrac{d}{2\delta}(\varphi(\bx+\delta\bw)-\varphi(\bx-\delta\bw))\bw}^2
% &\lesssim dC_\varphi^2~.
% \end{align*}
% We first show that replacing the gradient estimator with the inexact evaluations $\tF(\cdot)$ leads to a biased gradient estimator of $\varphi$.

% \begin{lemma}[Inexact gradient oracle] \label{lem: inexact gradient}
% Suppose $|\varphi(\cdot)-\tF(\cdot)|\leq\zeta$. Denoting
% \begin{align*}
% \bg_\bx&=\tfrac{d}{2\delta}(\varphi(\bx+\delta\bw)-\varphi(\bx-\delta\bw))\bw
% \\
% \tbg_\bx&=\tfrac{d}{2\delta}(\tF(\bx+\delta\bw)-\tF(\bx-\delta\bw))\bw
% \end{align*}
% it holds that
% \[
% \E_{\bw\sim\S^{d-1}}\norm{\bg_\bx-\tbg_\bx}\leq\frac{\zeta d}{\delta}~,
% ~~~~
% \E_{\bw\sim\S^{d-1}}\norm{\tbg_\bx}^2\lesssim \frac{\zeta^2 d^2}{\delta^2}+C_{\varphi}^2d~.
% \]
% \end{lemma}

% \begin{proof}
% For the first bound, we have
% \[
% \E_{\bw\sim\S^{d-1}}\norm{\bg_\bx-\tbg_\bx}
% \leq \frac{d}{2\delta}(2\zeta)\E_{\bw}\norm{\bw}
% =\frac{\zeta d}{\delta}~,
% \]
% while for the second bound
% \[
% \E_{\bw\sim\S^{d-1}}\norm{\tbg_\bx}^2
% =\E_{\bw}\norm{\tbg_\bx-\bg_\bx+\bg_\bx}^2
% \leq 2\E_{\bw}\norm{\tbg_\bx-\bg_\bx}^2+\E_{\bw}\norm{\bg_\bx}^2
% \lesssim \frac{d^2}{\delta^2}\cdot\zeta^2+C_{\varphi}^2d~.
% \]
% \end{proof}
We recall  \cref{thm:Lipschitz-min-with-inexact-grad-oracle} below to keep this section self-contained. 

\thmLipscMinWithInexactGradOracle*


Our analysis is inspired by the reduction from online learning to nonconvex optimization given by \cite{cutkosky2023optimal}.
To that end, we start by proving a seemingly unrelated result, asserting that online gradient descent minimizes the regret with respect to inexact evaluations. Recalling standard definitions from online learning, given a sequence of linear losses $\ell_m(\cdot)=\inner{g_m,\cdot}$, if an algorithm chooses ${\Delta}_1,\dots,{\Delta}_M$ we denote the regret with respect to $u$ as
\[
\mathrm{Reg}_M(u):=\sum_{m=1}^{M}\inner{g_m,{\Delta}_m-u}.
\]
Consider an update rule according to online projected \emph{inexact} gradient descent:
\[
\Delta_{m+1}:=\mathrm{clip}_{D}(\Delta_{m}-\eta_m\tbg_m).
\]

\begin{lemma}[Inexact Online Gradient Descent] \label{lem: inexact OGD}
In the setting above, suppose that $(\tbg_m)_{m=1}^{M}$ are possibly randomized vectors, such that
$\E\norm{\tbg_m-g_m}\leq\alpha$ and $\E\norm{\tbg_m}^2\leq\widetilde{G}^2$ for all $m\in[M]$.
Then for any $\norm{u}\leq D$ it holds that
\[
\E\left[\mathrm{Reg}_M(u)\right]\leq \frac{D^2}{\eta_M}+\widetilde{G}^2\sum_{m=1}^{M}\eta_m+ \alpha DM~,
\]
where the expectation is with respect to the (possible) randomness of $(\tbg_m)_{m=1}^{M}$.
In particular, setting $\eta_m\equiv\frac{D}{\widetilde{G}\sqrt{M}}$ yields
\[
\E\left[\mathrm{Reg}_M(u)\right]\lesssim D\widetilde{G}\sqrt{M}+\alpha DM~.
\]
\end{lemma}

\begin{proof}
For any $m\in[M]:$
\begin{align*}
\norm{\Delta_{m+1}-u}^2
&=\norm{\mathrm{clip}_{D}(\Delta_{m}-\eta_m\tbg_m)-u}^2
\\&\leq \norm{\Delta_{m}-\eta_m\tbg_m-u}^2
=\norm{\Delta_{m}-u}^2+\eta_m^2\norm{\tbg_m}^2-2\eta_m\inner{\Delta_{m}-u,\tbg_m}~,
\end{align*}
thus
\[
\inner{\tbg_m,\Delta_m-u}
\leq \frac{\norm{\Delta_m-u}^2-\norm{\Delta_{m+1}-u}^2}{2\eta_m}+\frac{\eta_m}{2}\norm{\tbg_m}^2~,
\]
from which we get that
\begin{align*}
\E\inner{g_m,\Delta_m-u}
&=\E\inner{\tbg_m,\Delta_m-u}+
\E\inner{g_m-\tbg_m,\Delta_m-u}
\\
&\leq\frac{\norm{\Delta_m-u}^2-\norm{\Delta_{m+1}-u}^2}{2\eta_m}+\frac{\eta_m}{2}\E\norm{\tbg_m}^2+\E\norm{g_m-\tbg_m}\cdot \norm{\Delta_m-u}
\\&\leq\frac{\norm{\Delta_m-u}^2-\norm{\Delta_{m+1}-u}^2}{2\eta_m}+\frac{\eta_m}{2}\widetilde{G}^2+\alpha D~.
\end{align*}
Summing over $m\in[M]$, we see that
\begin{align*}
\E\left[\mathrm{Reg}_M(u)\right]
&\leq \sum_{m=1}^{M}\norm{\Delta_m-u}^2\left(\frac{1}{\eta_m}-\frac{1}{\eta_{m-1}}\right)+\frac{\widetilde{G}^2}{2}\sum_{m=1}^{M}\eta_m+M\alpha D
\\
&\leq \frac{D^2}{\eta_M}+\widetilde{G}^2\sum_{m=1}^{M}\eta_m+ \alpha DM~.
\end{align*}
The simplification for $\eta_m\equiv\frac{D}{\widetilde{G}\sqrt{M}}$ readily follows.
\end{proof}


We are now ready to analyze \cref{alg: OIGRM} in the inexact gradient setting.
% We denote $\alpha=\frac{\zeta d}{\delta},~\widetilde{G}=\sqrt{\frac{\zeta^2 d^2}{\delta^2}+C_{\varphi}^2d}$. By \citet[Lemma 4]{kornowski2023algorithm}, it suffices to show that the algorithm produces a $(\delta,\epsilon)$-stationary point of $\varphi_{\delta}$. To that end,
\begin{proof}[Proof of \cref{thm:Lipschitz-min-with-inexact-grad-oracle}]
Since \cref{alg: OIGRM}  has $x_t=x_{t-1}+\Delta_{t}$, we have
\begin{align*}
F(x_t)-F(x_{t-1})
&=\int_{0}^{1}\inner{\nabla F(x_{t-1}+s\Delta_t),\Delta_t}ds
\\
&=\E_{s_t\sim\Unif[0,1]}\left[\langle\nabla F(x_{t-1}+s_t{\Delta}_t),\Delta_t\rangle\right]
\\&=\E\left[\inner{\nabla F(z_{t}),\Delta_t}\right]~.
% \\&=\E\left[\inner{\tbg_{t},\bDelta_t}\right]+\E\left[\inner{\nabla\varphi_{\delta}(\bz_{t})-\tbg_{t},\bDelta_t}\right]
% \\&\leq\E\left[\inner{\tbg_{t},\bDelta_t}\right]+\E\left[\norm{\nabla\varphi_{\delta}(\bz_{t})-\tbg_{t}}\cdot\norm{\bDelta_t}\right]
% \\& \leq\E\left[\inner{\tbg_{t},\bDelta_t}\right]+D\alpha~.
\end{align*}
% where the last inequality follows from Lemma~\ref{lem: inexact gradient}.
By summing over $t\in[T]=[K\times M]$, we get for any fixed sequence $u_1,\dots,u_K\in\reals^d:$
\begin{align*}
    \inf F\leq F(x_T)
    &\leq F(x_0)+\sum_{t=1}^{T}\E\left[\inner{\nabla F(z_{t}),\Delta_t}\right]
    \\&=F(x_0)+\sum_{k=1}^{K}\sum_{m=1}^{M}\E\left[\inner{\nabla F(z_{(k-1)M+m}),\Delta_{(k-1)M+m}-u_k}\right]\\
    &~~~~+\sum_{k=1}^{K}\sum_{m=1}^{M}\E\left[\inner{\nabla F(z_{(k-1)M+m}),u_k}\right]
    \\
    &\leq  F(x_0)+\sum_{k=1}^{K}\mathrm{Reg}_M(u_k)+\sum_{k=1}^{K}\sum_{m=1}^{M}\E\left[\inner{\nabla F(z_{(k-1)M+m}),u_k}\right]
    \\
    &\leq F(x_0)+KD\widetilde{G}\sqrt{M}+K\alpha DM+\sum_{k=1}^{K}\sum_{m=1}^{M}\E\left[\inner{\nabla F(z_{(k-1)M+m}),u_k}\right]
\end{align*}
where the last inequality follows from \cref{lem: inexact OGD}
for $\widetilde{G}=\sqrt{L^2+\alpha^2},~\eta=\frac{D}{\widetilde{G}\sqrt{M}}$, since 
$\norm{\tbg_t-\nabla F(z_t)}\leq\alpha$ (deterministically) for all $t\in[T]$ by assumption.
Letting $u_k:=-D\frac{\sum_{m=1}^{M}\nabla F(z_{(k-1)M+m})}{\norm{\sum_{m=1}^{M}\nabla F(z_{(k-1)M+m})}}$, rearranging and dividing by $DT=DKM$, we obtain
\begin{align}
\frac{1}{K}\sum_{k=1}^{K}\E\norm{\frac{1}{M}\sum_{m=1}^{M}\nabla F(z_{(k-1)M+m})}
&\leq \frac{F(x_0)-\inf F}{DT}+\frac{\widetilde{G}}{\sqrt{M}}+\alpha
\nonumber\\
&=\frac{F(x_0)-\inf F}{K\delta}
+\frac{\widetilde{G}}{\sqrt{M}}+\alpha~.
\label{eq: bound eps_first}
\end{align}
Finally, note that for all $k\in[K],m\in[M]:\norm{z_{(k-1)M+m}-\overline{x}_{k}}\leq M D\leq\delta$, therefore
$\nabla F(z_{(k-1)M+m})\in\partial_\delta F(\overline{x}_{k})$. Invoking the convexity of the Goldstein subdifferential, we see that
\[
\frac{1}{M}\sum_{m=1}^{M}\nabla F(z_{(k-1)M+m})\in\partial_\delta  F(\overline{x}_{k})~,
\]
thus it suffices to bound the first two summands on the right-hand side in \cref{eq: bound eps_first} by $\epsilon$ in order to finish the proof. This happens as long as $
\frac{F(x_0)-\inf F}{K\delta}\leq\frac{\epsilon}{2}$ and $\frac{\widetilde{G}}{\sqrt{M}}\leq\frac{\epsilon}{2}$. These are  equivalent to $ K\geq \frac{2(F(x_0)-\inf F)}{\delta\epsilon}$ and $M\geq\frac{4\widetilde{G}^2}{\epsilon^2}$, 
% \begin{align*}
% \frac{F(x_0)-\inf F}{K\delta}\leq\frac{\epsilon}{2}
% &\iff K\geq \frac{2(F(x_0)-\inf F)}{\delta\epsilon}
% \\
% \frac{\widetilde{G}}{\sqrt{M}}\leq\frac{\epsilon}{2}
% &\iff M\geq\frac{4\widetilde{G}^2}{\epsilon^2}~,
% \end{align*}
which results in \[T=KM=O\left(\frac{F(x_0)-\inf F}{\delta\epsilon}\cdot \frac{L^2+\alpha^2}{\epsilon^2}\right)=O\left(\frac{(F(x_0)-\inf F)L^2}{\delta\epsilon^3}\right),\] completing the proof.
\end{proof}










\subsection{An implementation-friendly algorithm and its analysis}
\begin{algorithm}[h]
\begin{algorithmic}[1]\caption{Perturbed Inexact GD}\label{alg: PIGD}
\State \textbf{Input:}
Inexact gradient oracle $\widetilde{\nabla}F:\reals^d\to\reals^d$, initialization $x_0\in\reals^d$, spatial parameter $\delta>0$, step size $\eta>0$, iteration budget $T\in\NN$.
\For{$t=0,\dots,T-1$}
\State Sample $w_t\sim\Unif(\S^{d-1})$
\State $\tbg_t=\widetilde{\nabla}F(x_{t}+\delta\cdot w_t)$
\State $x_{t+1}
=x_t-\eta\tbg_t
$
\EndFor
\State \textbf{Output:} $x^{\out}\sim\mathrm{Unif}\{x_0,\dots,x_{T-1}\}$. 
\end{algorithmic}
\end{algorithm}


\begin{theorem}\label{thm:practical_Lipschitz-min-with-inexact-grad-oracle}
Suppose $F:\reals^d\to\reals$ is $L$-Lipschitz, 
and that $\|\widetilde{\nabla} F(\cdot)-\nabla F(\cdot)\|\leq\alpha$. 
Then running \cref{alg: PIGD} with
$\eta=\Theta\left(\frac{{((F(x_0)-\inf F)+\delta L)^{1/2}\delta^{1/2}}}{{T^{1/2} L^{1/2}{d}^{1/4}(\alpha+L)}}\right)$
outputs a point $x^{\out}$ such that $\E[\mathrm{dist}(0,{\partial}_\delta F(x^{\out}))]\leq\epsilon+\sqrt{\alpha L}$, with \[T=O\left(\frac{(F(x_0)-\inf F+\delta L) L^3 \sqrt{d}}{\delta\epsilon^4}\right) \text{ 
calls to } \widetilde{\nabla}F(\cdot).\] 

\end{theorem}

\begin{proof}

Throughout the proof we denote $z_t=x_{t}+\delta\cdot w_t$.
Since $F$ is $L$-Lipschitz, $F_\delta(x):=\E_{w\sim\mathrm{Unif}(\S^{d-1})}[F(x+\delta\cdot w)]$ is $L$-Lipschitz and $O(L\sqrt{d}/\delta)$-smooth. By smoothness we get
\begin{align*}
F_{\delta}(x_{t+1})-F_{\delta}(x_{t})
&\leq \inner{\nabla F_{\delta}(x_t),x_{t+1}-x_{t}}+O\left(\frac{L\sqrt{d}}{\delta}\right)\cdot\norm{x_{t+1}-x_{t}}^2
\\
&=-\eta\inner{\nabla F_{\delta}(x_t),\tbg_t}+O\left(\frac{\eta^2 L\sqrt{d}}{\delta}\right)\cdot\norm{\tbg_t}^2
\\
&=-\eta\inner{\nabla F_{\delta}(x_t),\nabla F(z_t)}-\eta\inner{\nabla F_{\delta}(x_t),\tbg_t-\nabla F(z_t)}+O\left(\frac{\eta^2 L\sqrt{d}}{\delta}\right)\cdot\norm{\tbg_t}^2~.
\end{align*}


Noting that $\E[\nabla F(z_t)]=\nabla F_\delta(x_t)$
and that
$\norm{\tbg_t}\leq
\norm{\tbg_t-\nabla F(z_t)}+\norm{\nabla F(z_t)}
\leq \alpha+L$, we see that
\begin{align*}
    \E[F_{\delta}(x_{t+1})-F_{\delta}(x_{t})]
    \leq -\eta \E\norm{\nabla F_\delta (x_t)}^2+\eta L\alpha+O\left(\frac{\eta^2 L\sqrt{d}}{\delta}(\alpha+L)^2\right)~,
\end{align*}
which implies
\[
\E\norm{\nabla F_\delta (x_t)}^2 \leq \frac{\E[F_\delta(x_t)]-\E[F_\delta(x_{t+1})]}{\eta}+L\alpha+O\left(\frac{\eta L\sqrt{d}(\alpha+L)^2}{\delta}\right)~.
\]
Averaging over $t=0,\dots,T-1$ 
and noting that $F_\delta(x_0)-\inf F_\delta \leq (F(x_0)-\inf F) +\delta L$ results in
\[
\E\norm{\nabla F_\delta (x^{\out})}^2=\frac{1}{T}\sum_{t=0}^{T-1}E\norm{\nabla F_\delta (x_t)}^2
\leq \frac{(F(x_0)-\inf F)+\delta L}{\eta T}+L\alpha+O\left(\frac{\eta L\sqrt{d}(\alpha+L)^2}{\delta}\right)~.
\]
By Jensen's inequality and the sub-additivity of the square root,
\[
\E\norm{\nabla F_\delta (x^{\out})} \leq \sqrt{\frac{(F(x_0)-\inf F)+\delta L}{\eta T}}+\sqrt{L\alpha}+O\left(\sqrt{\frac{\eta L\sqrt{d}(\alpha+L)^2}{\delta}}\right)~.
\]
Setting $\eta=\frac{\sqrt{((F(x_0)-\inf F)+\delta L)\delta}}{\sqrt{T L\sqrt{d}(\alpha+L)^2}}$ yields the final bound
\[
\E\norm{\nabla F_\delta (x^{\out})} \lesssim
\frac{((F(x_0)-\inf F)+\delta L)^{1/4}L^{1/4}d^{1/8}(\alpha+L)^{1/2}}{\delta^{1/4}T^{1/4}}+\sqrt{L\alpha}~,
\]
and the first summand is bounded by $\epsilon$ for $T=O\left(\frac{((F(x_0)-\inf F)+\delta L) L \sqrt{d}(L+\alpha)^2}{\delta\epsilon^4}\right)$.



\end{proof}












\section{Reformulation equivalence}\label{appendix:reformulation-equivalence}
\begin{restatable}[Reformulation equivalence]{theorem}{reformulation}\label{thm:reformulation_equivalence}
When $\lambda^*$ matches to an optimal dual solution to the lower level problem $y^* = \arg\min_y g(x,y) ~\text{s.t.} ~h(x,y) \leq 0$, we show that for each $x$, the reformulation has the same feasible region of $y$. 
\end{restatable}
\begin{proof}
 We first show that lower-level feasibility implies feasibility of the reformulated problem.     
    Let $y^*, \lambda^* = \min\limits_y \max\limits_{\beta \geq 0} g(x,y) + \beta^\top h(x,y)$ be the primal and the dual solution to the lower level problem with parameter $x$.
    We can verify that $y^*$ satisfies all the constraints in the reformulation problem. The feasibility condition $h(x,y^*)$ is automatically satisfied.
    We just need to check:
    % \pswt{I think we had discussed this before, but I forgot: since $h\leq 0$ is a scalar constraint, wouldn't $\lambda^*$ be a scalar? If we are allowing $h$ to be vector-valued, then what do we mean by convexity/smoothness of $h$?} <- ok, but need to change/check dimensions/order of operations everywhere
    \begin{align*}
        g^*(x) & \coloneqq  \min\limits_\theta g(x,\theta) + (\lambda^*)^\top h(x,\theta) \\
        & = g(x,y^*) + (\lambda^*)^\top h(x,y^*).\numberthis\label[eq]{eq:g_gamma_star_equals_g_plus_gamma_h}
    \end{align*}
    Therefore, $x, y^*$ is a feasible point to the reformulation problem.
    
    We now show the other direction, i.e., that feasibility of the reformulaed problem implies that of the lower-level problem. 
    % (Reformulation feasibility $\Longrightarrow$ lower level feasibility)
    Given $\lambda^*$, let us assume $y$ satisfies $g(x,y) \leq g^*_{\lambda^*}(x)$ and $h(x,y) \leq 0$.
    On the other hand, assume $y^*, \lambda^* = \min\limits_y \max\limits_{\beta \geq 0} g(x,y) + \beta^\top h(x,y)$ be the primal and the dual solution.
    We can show that:
    \begin{align}
        g(x,y) + (\lambda^*)^\top h(x,y) \leq g^*(x) \coloneqq \min_\theta g(x,\theta) + (\lambda^*)^\top h(x,\theta). 
    \end{align}
    By the strong convexity of $g + (\lambda^*)^\top h$, we know that $y$ matches to the unique minimum $y^*$, which implies that $y = y^*$ is also a feasible point to the original bilevel problem.
\end{proof}

%\pswt{note: just lower-casing all titles since that's stated explcitily in the NeurIPS CFP}
\section{Active constraints in differentiable optimization}\label{sec:inactive-constraints-in-differentiable-optimization}
By computing the derivative of the KKT conditions in \cref{sec:differentiable-optimization}, we get:
\begin{align}
    (\nabla^2_{yx} g + (\lambda^*)^\top \nabla_{yx}^2 h) + (\nabla^2_{yy} g + (\lambda^*)^\top \nabla_{yy}^2 h) \frac{dy^*}{dx} + (\nabla_y h)^\top \frac{d\lambda^*}{dx} &= 0 \label{eqn:appendix-KKT1} \\
    \text{diag}(\lambda^*) \nabla_x h + \text{diag}(\lambda^*) \nabla_y h \frac{dy^*}{dx} + \text{diag}(h) \frac{d\lambda^*}{dx} &= 0. \label{eqn:appendix-KKT2}
\end{align}

Let $\mathcal{I} = \{ i \in [d_h] | h(x,y^*)_i = 0, \lambda^*_i > 0 \}$ be the set of active constraints with positive dual solution, and $\mathcal{I}_1 = \{ i | h(x,y^*)_i \neq 0 \}$ be the set of inactive constraints and $\mathcal{I}_2 = \{ i | h(x,y^*)_i = 0, \lambda^*_i = 0 \}$. We know that $\bar{\mathcal{I}} = \mathcal{I}_1 \cup \mathcal{I}_2$. For each $i \in \mathcal{I}_1$, due to complementary slackness, we know that $\lambda^*_i = 0$. 

For $i \in \mathcal{I}_1$ in \cref{eqn:appendix-KKT1}, we have $\lambda^*_i \nabla_x h(x,y^*)_i + \lambda^*_i \nabla_y h(x,y^*)_i \frac{dy^*}{dx} + h(x,y^*)_i \frac{d \lambda^*_i}{dx} = 0$, which implies $h(x,y^*)_i \frac{d \lambda^*_i}{dx} = 0$ because  $\lambda^*_i = 0$. This in turn implies $\frac{d \lambda^*_i}{dx} = 0 $ because $h(x,y^*)_i < 0$.
% \begin{align}
%     &  \nonumber \\
%     \Longrightarrow \quad &  \quad \quad \text{(because)} \nonumber \\
%     \Longrightarrow \quad &  \quad \quad \quad \quad \quad ~ ~ \text{(because )} \nonumber
% \end{align}
That means the dual variable $\lambda^*_i = 0$ and has zero gradient $\frac{d \lambda^*_i}{dx} = 0$ for any index $i \in \mathcal{I}_1$.
Therefore, we can remove row $i \in \mathcal{I}_1$ in \cref{eqn:appendix-KKT2} and obtain $\lambda^*_i = 0$ and $\frac{d \lambda^*_i}{dx} = 0$.

For $i \in \mathcal{I}_2$, the KKT condition in \cref{eqn:appendix-KKT2} is  degenerate. Therefore, $\frac{d \lambda^*_i}{d x}$ can be arbitrary, i.e., non-differentiable.
As a subgradient choice, we can set $\frac{d \lambda^*_i}{d x} = 0$ for such $i$. 
This choice will also eliminate its impact on the KKT condition in \cref{eqn:appendix-KKT1} because $\frac{d \lambda^*_i}{d x}$ is set to be $0$.
By this choice of subgradient, we can also remove row $i \in \mathcal{I}_2$ \cref{eqn:appendix-KKT2}.

Thus \cref{eqn:appendix-KKT2} can be written as the following set of equations, for  $h_\mathcal{I} = [h_i]_{i \in \mathcal{I}}$ and $\lambda^*_\mathcal{I} = [\lambda^*_i]_{i\in \mathcal{I}}$:
\begin{align}
    & \text{diag}(\lambda^*) \nabla_x h_\mathcal{I} + \text{diag}(\lambda^*_\mathcal{I}) \nabla_y h_\mathcal{I} \frac{dy^*}{dx} + \text{diag}(h_\mathcal{I}) \frac{d\lambda^*_\mathcal{I}}{dx} = 0 \nonumber \\
    \Longrightarrow \quad & \text{diag}(\lambda^*) \nabla_x h_\mathcal{I} + \text{diag}(\lambda^*_\mathcal{I}) \nabla_y h_\mathcal{I} \frac{dy^*}{dx} = 0 \quad \text{(due to $h_\mathcal{I}(x,y^*) = 0$)}. \label{eqn:appendix-new-KKT2}
\end{align}

In \cref{eqn:appendix-KKT1}, due to $\frac{d \lambda^*_i}{dx} = 0$ for all $i \in \bar{\mathcal{I}}$, we can remove $\frac{d \lambda^*_i}{dx} ~\forall i \in \bar{\mathcal{I}}$ in \cref{eqn:appendix-KKT1} by:
\begin{align}
    0 = & ~ (\nabla^2_{yx} g + (\lambda^*)^\top \nabla_{yx}^2 h) + (\nabla^2_{yy} g + (\lambda^*)^\top \nabla_{yy}^2 h) \frac{dy^*}{dx} + (\nabla_y h)^\top \frac{d\lambda^*}{dx} \nonumber \\
    = & ~ (\nabla^2_{yx} g + (\lambda^*)^\top \nabla_{yx}^2 h) + (\nabla^2_{yy} g + (\lambda^*)^\top \nabla_{yy}^2 h) \frac{dy^*}{dx} + (\nabla_y h_\mathcal{I})^\top \frac{d\lambda^*_\mathcal{I}}{dx}. 
    \label{eqn:appendix-new-KKT1} 
\end{align}

Combining \cref{eqn:appendix-new-KKT1} and \cref{eqn:appendix-new-KKT2}, we get:
\begin{align*}
    (\nabla^2_{yx} g + (\lambda^*)^\top \nabla_{yx}^2 h) + (\nabla^2_{yy} g + (\lambda^*)^\top \nabla_{yy}^2 h) \frac{dy^*}{dx} + (\nabla_y h_\mathcal{I})^\top \frac{d\lambda^*_\mathcal{I}}{dx} &= 0 \\
    \text{diag}(\lambda^*) \nabla_x h_\mathcal{I} + \text{diag}(\lambda^*_\mathcal{I}) \nabla_y h_\mathcal{I} \frac{dy^*}{dx} &= 0, 
\end{align*}
which can be written in its matrix form:
\begin{align}\label{eqn:kkt-system_appendix}
\begin{bmatrix}
\nabla^2_{yy} g + (\lambda^*)^\top \nabla_{yy}^2 h & \nabla_y h_\mathcal{I}^\top \\
\text{diag}(\lambda^*_\mathcal{I}) \nabla_y h_\mathcal{I} & 0
\end{bmatrix}
\begin{bmatrix}
    \frac{dy^*}{dx} \\
    \frac{d\lambda^*_\mathcal{I}}{dx}
\end{bmatrix}
= 
-
\begin{bmatrix}
    \nabla^2_{yx} g + (\lambda^*)^\top \nabla_{yx}^2 h \\
    \text{diag}(\lambda^*_\mathcal{I}) \nabla_x h_\mathcal{I}
\end{bmatrix}
\end{align}
This concludes the derivation of the derivative of constrained optimization in \cref{eqn:kkt-system}.


\section{Inequality case: bounds on primal solution error and constraint violation}
\solutionApproximation*

\begin{proof}
We first provide the claimed bound on $\|y^*_{\alpha_1, \alpha_2} - y^*(x)\|$. 

\noindent\textbf{Part 1: Bound on the convergence of $y$.}

Since $y_{\lambda^*,\boldsymbol{\alpha}}^*$ minimizes $\mathcal{L}_{\boldsymbol{\alpha}, \lambda^*}(x,y)$, the first-order condition gives us:
% \pswt{change  gradients to subgradients where required (e.g., for $\mathcal{L}$)}
\begin{align*}
    0 = \nabla_y \mathcal{L}_{\boldsymbol{\alpha}, \lambda^*}(x,y_{\lambda^*,\boldsymbol{\alpha}}^*). 
    % = \nabla_y f(x,y^*_{\alpha}) + \alpha_1(\nabla_y g(x,y_{\lambda^*,\boldsymbol{\alpha}}^*) + (\lambda^*)^\top \nabla_y h(x,y_{\lambda^*,\boldsymbol{\alpha}}^*)) + \alpha_2 h_\mathcal{I}(x,y_{\lambda^*,\boldsymbol{\alpha}}^*)^\top \nabla_y h(x,y_{\lambda^*,\boldsymbol{\alpha}}^*) \pswt{we do not need this expansion for this proof.}
\end{align*}
Similarly, we can compute the gradient of $\mathcal{L}_{\boldsymbol{\alpha}, \lambda^*}(x,y)$ at $y^*$: 
% \pswt{Check: is $\lambda^*$ a scalar or a vector?}
\begin{align*}
    \nabla_y \mathcal{L}_{\alpha}(x,y^*) & = \nabla_y f(x,y^*) + \alpha_1(\nabla_y g(x,y^*) + (\lambda^*)^\top \nabla_y h(x,y^*)) + \alpha_2 \nabla_y h_\mathcal{I}(x,y^*)^\top h_\mathcal{I}(x,y^*)  \\
    & = \nabla_y f(x,y^*),
\end{align*} 
% \pswt{Also check: is $h(x,y^*)$ above a scalar or vector? I'm not seeing the dimensions of $\nabla_y g(x,y^*)$ and $(\lambda^*)^\top \nabla_y h(x, y^*)$ matching up?}
where the second step is due to the property of the primal and dual solution: $\nabla_y g(x,y^*) + (\lambda^*)^\top \nabla_y h(x,y^*) = 0$ by the stationarity condition in the KKT conditions, and by definition of the active constraints $h_\mathcal{I}$ where the optimal $y^*$ must have $h_\mathcal{I}(x,y^*) = 0$.

Since, for a sufficiently large $\alpha_1$, the penalty function is $\alpha_1 \mu_g - L_f \geq \frac{\alpha_1 \mu_g}{2}$  strongly convex in $y$, we have:
\begin{align*}
    \frac{\alpha_1 \mu_g}{2} \norm{y^* - y_{\lambda^*,\boldsymbol{\alpha}}^*} \leq \norm{\nabla_y\mathcal{L}_{\boldsymbol{\alpha}, \lambda^*}(x,y^*) - \nabla_y\mathcal{L}_{\boldsymbol{\alpha}, \lambda^*}(x,y_{\lambda^*,\boldsymbol{\alpha}}^*)} = \norm{\nabla_y f(x,y^*)} \leq L_f.
\end{align*}
Therefore, upon rearranging the terms,  we obtain the claimed bound:
\begin{align*}
    \norm{y^* - y^*_{\boldsymbol{\alpha}, \lambda^*}} \leq \frac{2 L_f}{\alpha_1 \mu_g}.
\end{align*}

\noindent\textbf{Part 2: bound on the constraint violation.}

When we plug $y^*$ into \cref{eqn:penalty-lagrangian}, we get:
\begin{align*}
    \mathcal{L}_{\boldsymbol{\alpha}, \lambda^*}(x,y^*) & = f(x,y^*) + \alpha_1 (g(x,y^*) + (\lambda^*)^\top h(x,y^*) - g^*_{\lambda^*}(x)) + \frac{\alpha_2}{2} \norm{h_\mathcal{I}(x,y^*)}^2 = f(x,y^*).
\end{align*}
Plugging in $y^*_{\boldsymbol{\alpha},\lambda^*}$, we may obtain: 
\begin{align*}
    \mathcal{L}_{\boldsymbol{\alpha}, \lambda^*}(x,y_{\lambda^*,\boldsymbol{\alpha}}^*) & = f(x,y_{\lambda^*,\boldsymbol{\alpha}}^*) + \alpha_1 (g(x,y_{\lambda^*,\boldsymbol{\alpha}}^*) + (\lambda^*)^\top h(x,y_{\lambda^*,\boldsymbol{\alpha}}^*) - g^*(x)) + \frac{\alpha_2}{2} \norm{h_\mathcal{I}(x,y_{\lambda^*,\boldsymbol{\alpha}}^*)}^2 \\
    & = f(x,y_{\lambda^*,\boldsymbol{\alpha}}^*) + \alpha_1 (  g(x,y_{\lambda^*,\boldsymbol{\alpha}}^*) + (\lambda^*)^\top h(x,y_{\lambda^*,\boldsymbol{\alpha}}^*) - g(x,y^*) - (\lambda^*)^\top h(x,y^*) ) \\
    & \qquad + \frac{\alpha_2}{2} \norm{h_\mathcal{I}(x,y_{\lambda^*,\boldsymbol{\alpha}}^*)}^2 \\
    & \geq f(x,y_{\lambda^*,\boldsymbol{\alpha}}^*) + \alpha_1 \mu_g \norm{y^* - y_{\lambda^*,\boldsymbol{\alpha}}^*}^2 + \frac{\alpha_2}{2} \norm{h_\mathcal{I}(x,y_{\lambda^*,\boldsymbol{\alpha}}^*)}^2,
\end{align*} where we used the strong convexity (with respect to $y$) of $g(x,y)+(\lambda^*)^\top h(x,y)$ and the optimality of $y^*$ for $g(x,y)+(\lambda^*)^\top h(x,y)$. 
% \pswt{Kai: we seem to be using that the strong convexity of $g+(\lambda^*)^\top h$ is $\mu_g/2$; is it obvious why this is the case (i.e., no dependence on $\lambda^*$)?}
By the optimality of $y_{\lambda^*,\boldsymbol{\alpha}}^*$ for $\mathcal{L}_{\boldsymbol{\alpha}, \lambda^*}$, we know that 
\begin{align*}
    f(x,y^*) = \mathcal{L}_{\boldsymbol{\alpha}, \lambda^*}(x,y^*) \geq \mathcal{L}_{\boldsymbol{\alpha}, \lambda^*}(x,y_{\lambda^*,\boldsymbol{\alpha}}^*) \geq f(x,y_{\lambda^*,\boldsymbol{\alpha}}^*) + \alpha_1 \mu_g \norm{y^* - y_{\lambda^*,\boldsymbol{\alpha}}^*}^2 + \frac{\alpha_2}{2} \norm{h_\mathcal{I}(x,y_{\lambda^*,\boldsymbol{\alpha}}^*)}^2.
\end{align*}
Therefore, by the Lipschitzness of the function $f$ in terms of $y$, and the bound $\|y^* - y_{\lambda^*,\boldsymbol{\alpha}}^*\| \leq \frac{2L_f}{\alpha_1 \mu_g}$, we know that:\begin{align*}
    \frac{\alpha_2}{2} \norm{h_\mathcal{I}(x,y_{\lambda^*,\boldsymbol{\alpha}}^*)}^2 & \leq f(x,y^*) - f(x,y_{\lambda^*,\boldsymbol{\alpha}}^*) - \alpha_1 \mu_g \norm{y^* - y_{\lambda^*,\boldsymbol{\alpha}}^*}^2 \\
    & \leq L_f \norm{y^* - y_{\lambda^*,\boldsymbol{\alpha}}^*} - \alpha_1 \mu_g \norm{y^* - y_{\lambda^*,\boldsymbol{\alpha}}^*}^2 \\
    & \leq L_f \norm{y^* - y_{\lambda^*,\boldsymbol{\alpha}}^*} \\
    & = O({\alpha_1^{-1}}). 
\end{align*}
Rearranging terms then gives the claimed bound. 
% Thus, we can then show:
% \begin{align}
%     \norm{h_\mathcal{I}(x,y_{\lambda^*,\boldsymbol{\alpha}}^*)} \leq O(\frac{1}{\sqrt{\alpha_1 \alpha_2}})
% \end{align}
\end{proof}
The bound on the constraint violation in \cref{{thm:solution-bound}} is an important step in the following theorem.

\section{Proof of \cref{thm:diff_in_hypergrad_and_gradLagr}: gradient approximation for inequality constraints}\label{appendix:proof-of-inexact-gradient}
\gradientApproximation*
\begin{proof}
First, we recall \cref{eqn:penalty-lagrangian} here: \[\mathcal{L}_{\lambda^*,\boldsymbol{\alpha}}(x,y) = f(x,y) + \alpha_1 \left( g(x,y) + (\lambda^*)^\top h(x,y) - g^*(x)  \right) + \frac{\alpha_2}{2} \norm{h_\mathcal{I}(x,y)}^2.\] Next, recall from \cref{eq:g_gamma_star_equals_g_plus_gamma_h}, we can express $g^*(x)=g(x,y^*)+(\lambda^*)^\top h(x,y^*)$, which we use in the first step below: 
% \pswt{Some justification for $\lambda^*$ not being differentiated w.r.t. $x$? Or, would we get an extra term of the form $\alpha_1 \nabla_x \lambda^*(x) \cdot h(x, \ysl)$?} \kai{Why we don't need $\nabla_x \lambda^*(x) \cdot h(x, \ysl)$? For active constraints $h_i(x,y) = 0$, this product is $0$. For strictly inactive constraints $h_i(x,y) < 0$, we can show that $d \lambda^*_i(x) / dx = 0$ by the KKT system so it is also $0$. For $i$ such that $h_i(x,y) = 0$ and $\lambda^*_i = 0$, the KKT system degenerates (non-differentiable) but we can pick a valid subgradient such that $d \lambda^*_i(x) / dx = 0$, and thus its product is still $0$.} \pswt{Very cool; do we need to add this explanation, or just let it be?}
% \kai{Added a definition of $\mathcal{L}$ in Theorem 3.3} \pswt{Justification for $y^*$ being differentiable}

% \kai{I fixed the gradient notations based on our discussion. There is one counter-intuitive one where $\nabla_x F(x) = \nabla_x f(x,y^*) + \frac{d y^*}{d x}^\top \nabla_y f(x,y^*)$ as oppose to $\nabla_x F(x) = \nabla_x f(x,y^*) + \nabla_y f(x,y^*)^\top  \frac{d y^*}{d x}$ due to column v.s. row dimension (check their dimensions and it will be clear).} \pswt{I agree, $\nabla_x F(x) = \nabla_x f(x,y^*) + \frac{d y^*}{d x}^\top \nabla_y f(x,y^*)$; changed it to this throughout}
\begin{align}
    \footnotesize
    & \nabla_x F(x) - \frac{d}{dx} \mathcal{L}_{\lambda^*,\boldsymbol{\alpha}}(x,y_{\lambda^*,\alpha}^*) \nonumber \\
    = & \left( \nabla_x f(x,y^*) + \frac{d y^*}{d x}^\top \nabla_y f(x,y^*) \right) - \Biggl( \nabla_x f(x,y_{\lambda^*,\boldsymbol{\alpha}}^*) + \alpha_1 (\nabla_x g(x,y_{\lambda^*,\boldsymbol{\alpha}}^*) + \nabla_x h(x,y_{\lambda^*,\boldsymbol{\alpha}}^*)^\top \lambda^* \nonumber \\
    &  - \alpha_1(\nabla_x g(x,y^*) + \nabla_x h(x,y^*)^\top \lambda^*) 
    + \alpha_2
    \nabla_x h_\mathcal{I}(x,y_{\lambda^*,\boldsymbol{\alpha}}^*)^\top h_\mathcal{I}(x,y_{\lambda^*,\boldsymbol{\alpha}}^*)
\Biggl) \nonumber \\
    = & \nabla_x f(x,y^*) - \nabla_x f(x,y_{\lambda^*,\boldsymbol{\alpha}}^*) \label{eqn:f-difference} \\
    & + \frac{d y^*}{d x}^\top \nabla_y f(x,y^*) - \frac{d y^*}{d x}^\top \nabla_y f(x,y_{\lambda^*,\boldsymbol{\alpha}}^*) \label{eqn:df-difference} \\ 
    & + \frac{d y^*}{d x}^\top \nabla_y f(x,y_{\lambda^*,\boldsymbol{\alpha}}^*) - \underbrace{\alpha_1 \begin{bmatrix}
        \nabla^2_{yx} g + (\lambda^*)^\top \nabla_{yx}^2 h \\
        \text{diag}(\lambda^*_\mathcal{I}) \nabla_x h_\mathcal{I}
    \end{bmatrix}^\top  \begin{bmatrix}
        y_{\lambda^*,\boldsymbol{\alpha}}^* - y^*  \\
        0
    \end{bmatrix} }_{\text{added term 1}} \nonumber\\
    &\qquad- \underbrace{\alpha_2 \begin{bmatrix}
        \nabla^2_{yx} g + (\lambda^*)^\top \nabla_{yx}^2 h \lambda^* \\
        \text{diag}(\lambda^*_\mathcal{I}) \nabla_x h_\mathcal{I}
    \end{bmatrix}^\top \begin{bmatrix}
        0 \\
        \text{diag}(1/\lambda^*_\mathcal{I}) h_\mathcal{I}(x,y_{\lambda^*,\boldsymbol{\alpha}}^*) 
    \end{bmatrix}}_{\text{added term 2}} \label{eqn:df-and-added-term} \\
    & + \alpha_1 \Biggl( \nabla_x g(x,y^*) - \nabla_x g(x,y_{\lambda^*,\boldsymbol{\alpha}}^*) + \nabla_x h(x, y^*)^\top \lambda^* - \nabla_x h(x, y_{\lambda^*,\boldsymbol{\alpha}}^*)^\top \lambda^* \nonumber\\
    &\qquad+ \underbrace{\begin{bmatrix}
        \nabla^2_{yx} g + (\lambda^*)^\top \nabla_{yx}^2 h \\
        \text{diag}(\lambda^*_\mathcal{I}) \nabla_x h_\mathcal{I}
    \end{bmatrix}^\top \begin{bmatrix}
        y_{\lambda^*,\boldsymbol{\alpha}}^* - y^*  \\
        0 % h_\mathcal{I}(x,y_{\lambda^*,\boldsymbol{\alpha}}^*)
    \end{bmatrix}}_{\text{added term 1}}
    \Biggl)
    \label{eqn:dgdh-and-added-term} \\
    & - \alpha_2 \nabla_x h_\mathcal{I}(x,y_{\lambda^*,\boldsymbol{\alpha}}^*)^\top h_\mathcal{I}(x,y_{\lambda^*,\boldsymbol{\alpha}}^*) + \underbrace{\alpha_2 \begin{bmatrix}
        \nabla^2_{yx} g + (\lambda^*)^\top \nabla_{yx}^2 h \lambda^* \\
        \text{diag}(\lambda^*_\mathcal{I}) \nabla_x h_\mathcal{I}
    \end{bmatrix}^\top \begin{bmatrix}
        0 \\
        \text{diag}(1/\lambda^*_\mathcal{I}) h_\mathcal{I}(x,y_{\lambda^*,\boldsymbol{\alpha}}^*) 
    \end{bmatrix}}_{\text{added term 2}}. \label{eqn:dh2-difference} 
\end{align}
% \pswt{removed $MM^{-1}$ for now since it's not used in the above steps and can be introduced later as needed;}
According to \cref{eqn:kkt-system} and \cref{eqn:kkt-system_appendix}, we let 
% \pswt{Kai: would the $\lambda^*$ in the $H$ below need to be $\lambda^*_\mathcal{I}$?} \kai{Yes thank you!} 
\[H = \begin{bmatrix}
    \nabla^2_{yy} g + (\lambda^*)^\top \nabla_{yy}^2 h & \nabla_y h_\mathcal{I}^\top \\
    \text{diag}((\lambda^*_\mathcal{I}) \nabla_y h_\mathcal{I} & 0
\end{bmatrix},\] which is invertible by \cref{item:assumption_tangen_space} and by the fact that we remove all the inactive constraints.
% \pswt{note to self: come back to this claim} % , which may not be invertible. But the top left entry $\nabla^2_{yy} g + (\nabla_{yy}^2 h)^\top (\lambda^*$ is strongly convex and invertible.
% \pswt{For \cref{eqn:df-and-added-term}, should we restrict to $S$, since later we say that that's premultiplied by the $S$-restricted $M$ to give us $dy^*/dx$? The same comment applies to the other matrices involving $\nabla^2_{yx} g$.}\pswt{<--- minor comment, can look at it later}
We now bound the terms in \cref{eqn:f-difference}, \cref{eqn:df-difference}, \cref{eqn:df-and-added-term}, \cref{eqn:dgdh-and-added-term}, and \cref{eqn:dh2-difference}.


\noindent\textbf{Bounding \cref{eqn:f-difference} and \cref{eqn:df-difference}:}
\cref{eqn:f-difference} can be easily bounded by the smoothness of $f$ 
% \pswt{aren't we using a bound on $\nabla^2_{yx} f$ here? (rather than one on  $\nabla_{yy}^2 f$)} \kai{Updated: thanks for catching this!} 
in terms of $x$ and $y$, and the bound on $\|y^*- y_{(\lambda^*,\boldsymbol{\alpha}}^*\| \leq O({\alpha_1^{-1}})$ from \cref{thm:solution-bound}. Therefore, we know:
\begin{align*}
    \norm{\nabla_x f(x,y^*) - \nabla_x f(x,y_{(\lambda^*,\boldsymbol{\alpha}}^*)} \leq C_f \norm{y^* - y_{(\lambda^*,\boldsymbol{\alpha}}^*} \leq C_f\cdot O({\alpha_1^{-1}}). 
\end{align*}
Similarly, given \cref{item:assumption_safe_constraints}  by which $y^*(x)$ is $L_y$-Lipschitz in $x$, we have the bound $\norm{\frac{d y^*}{d x}} \leq L_y$. Therefore, \cref{eqn:df-difference} can be bounded by:
% \pswt{Note: we are using Lipschitzness of $y^*$ below.}
\begin{align*}
    \norm{\frac{dy^*}{dx}^\top \nabla_y f(x,y^*) - \frac{dy^*}{dx}^\top \nabla_y f(x,y_{(\lambda^*,\boldsymbol{\alpha}}^*)} \leq C_f \norm{\frac{dy^*}{dx}} \norm{y^* - y_{(\lambda^*,\boldsymbol{\alpha}}^*} \leq C_f L_y\cdot O({\alpha_1^{-1}}). 
\end{align*}
% where $\frac{dy^*}{dx}$ can be bounded by bounding \cref{eqn:dy*dx} in Section~\ref{sec:differentiable-optimization}.


\noindent\textbf{Bounding \cref{eqn:df-and-added-term}:} 
% \pswt{dimensions on LHS and RHS?} 


Using~\cref{eqn:kkt-system} to solve $\begin{bmatrix}
    \frac{d y^*}{d x} \\
    \frac{d \lambda^*}{d x}
\end{bmatrix} = -H^{-1} \begin{bmatrix}
    \nabla^2_{yx} g + (\lambda^*)^\top \nabla_{yx}^2 h \\
    \text{diag}(\lambda^*_\mathcal{I}) \nabla_x h_\mathcal{I}
\end{bmatrix}$, 
% \pswt{Kai: the RHS in the previous equation looks like it's for the concatenated vector of $dy/dx$ and $d\lambda^*/dx$; so $dy/dx$ is actually some indicator vector times the RHS we have here. (But I think it's fine in the RHS of the next equation because of the added zeroes)} \kai{Thanks for finding the typo! Just fixed it.} 
we can write:
\begin{align*}
    & \frac{d y^*}{d x}^\top \nabla_y f(x,y_{\lambda^*,\boldsymbol{\alpha}}^*) =  \begin{bmatrix}
        \nabla^2_{yx} g + (\lambda^*)^\top \nabla_{yx}^2 h \\
        \text{diag}(\lambda^*_\mathcal{I}) \nabla_x h_\mathcal{I}
    \end{bmatrix}^\top (H^{-1})^\top \begin{bmatrix}
        - \nabla_y f(x,y_{\lambda^*,\boldsymbol{\alpha}}^*) \\ 0
    \end{bmatrix}   \\
     &= -\frac{d y^*}{d x}^\top  \Biggl( \alpha_1 \begin{bmatrix}
        \nabla_y g(x,y_{\lambda^*,\boldsymbol{\alpha}}^*) + \nabla_y h(x,y_{\lambda^*,\boldsymbol{\alpha}}^*)^\top \lambda^* \\
        0
    \end{bmatrix} \\ 
    &\quad + 
    \alpha_2 \begin{bmatrix}
         \nabla_y h_\mathcal{I}(x,y_{\lambda^*,\boldsymbol{\alpha}}^*)^\top h_\mathcal{I}(x,y_{\lambda^*,\boldsymbol{\alpha}}^*) \\
        0
    \end{bmatrix}
    \Biggl), \numberthis\label{eqn:dfdy_dydx_expansion}
\end{align*}
where we use the optimality of $y_{\lambda^*,\boldsymbol{\alpha}}^*$ from \cref{{eq:def_y_lambda_star}}:
\begin{align*}\numberthis\label{eqn:y_lambda_star_optimality}
    & \nabla_y f(x,y_{\lambda^*,\boldsymbol{\alpha}}^*) + \alpha_1 \left( \nabla_y g(x,y_{\lambda^*,\boldsymbol{\alpha}}^*) + \nabla_y h(x,y_{\lambda^*,\boldsymbol{\alpha}}^*)^\top \lambda^* \right) \\
    &\quad+ \alpha_2 \nabla_y h_\mathcal{I}(x,y_{\lambda^*,\boldsymbol{\alpha}}^*)^\top h_\mathcal{I}(x,y_{\lambda^*,\boldsymbol{\alpha}}^*) = 0. 
\end{align*}
Further, recall that $H$ is non-degenerate by \cref{assumption:linEq_smoothness}, as a result of which, the added term 1 in \cref{eqn:df-and-added-term} can be modified as follows: 
% \pswt{Would there be a factor of $\textrm{diag}(\lambda^*)$ in the lower term of the first matrix on the RHS, since it's in $M$ as well?}<- ok, good
% \pswt{Also, would the terms in the first matrix in the RHS  require the $S$ subscript since they come from $M$?}\pswt{<--- minor comment, will look later}
\begin{align}
    \footnotesize
    & \begin{bmatrix}
        \nabla^2_{yx} g + (\lambda^*)^\top \nabla_{yx}^2 h \\
        \text{diag}(\lambda^*_\mathcal{I}) \nabla_x h_\mathcal{I}
    \end{bmatrix}^\top \begin{bmatrix}
        \alpha_1 (y_{\lambda^*,\boldsymbol{\alpha}}^* - y^*) \\
        0
    \end{bmatrix} \nonumber \\
    = & \begin{bmatrix}
        \nabla^2_{yx} g + (\lambda^*)^\top \nabla_{yx}^2 h \\
        \text{diag}(\lambda^*_\mathcal{I}) \nabla_x h_\mathcal{I}
    \end{bmatrix}^\top (H^{-1})^\top  H^\top  \begin{bmatrix}
        \alpha_1 (y_{\lambda^*,\boldsymbol{\alpha}}^* - y^*) \\
        0
    \end{bmatrix} \nonumber \\
    = & \alpha_1 
    \begin{bmatrix}
        \nabla^2_{yx} g + (\lambda^*)^\top \nabla_{yx}^2 h \\
        \text{diag}(\lambda^*_\mathcal{I}) \nabla_x h_\mathcal{I}
    \end{bmatrix}^\top (H^{-1})^\top \begin{bmatrix}
        (\nabla^2_{yy} g + (\lambda^*)^\top \nabla_{yy}^2 h)^\top (y_{\lambda^*,\boldsymbol{\alpha}}^* - y^*) \\
        \nabla_y h_\mathcal{I}(x,y^*)(y_{\lambda^*,\boldsymbol{\alpha}}^* - y^*)
    \end{bmatrix}. \label{eqn:added_term}
\end{align}

The added term 2 in \cref{eqn:df-and-added-term} can be expanded to: 
\begin{align}
    & \alpha_2 \begin{bmatrix}
        \nabla^2_{yx} g + (\lambda^*)^\top \nabla_{yx}^2 h \lambda^* \\
        \text{diag}(\lambda^*_\mathcal{I}) \nabla_x h_\mathcal{I}
    \end{bmatrix}^\top \begin{bmatrix}
        0 \\
        \text{diag}(1/\lambda^*_\mathcal{I}) h_\mathcal{I}(x,y_{\lambda^*,\boldsymbol{\alpha}}^*) 
    \end{bmatrix} \nonumber \\
    = & \alpha_2 \begin{bmatrix}
        \nabla^2_{yx} g + (\lambda^*)^\top \nabla_{yx}^2 h \lambda^* \\
        \text{diag}(\lambda^*_\mathcal{I}) \nabla_x h_\mathcal{I}
    \end{bmatrix}^\top (H^{-1})^\top H^\top  \begin{bmatrix}
        0 \\
        \text{diag}(1/\lambda^*_\mathcal{I}) h_\mathcal{I}(x,y_{\lambda^*,\boldsymbol{\alpha}}^*) 
    \end{bmatrix} \nonumber \\
    = & \alpha_2  \begin{bmatrix}
        \nabla^2_{yx} g + (\lambda^*)^\top \nabla_{yx}^2 h \lambda^* \\
        \text{diag}(\lambda^*_\mathcal{I}) \nabla_x h_\mathcal{I}
    \end{bmatrix}^\top (H^{-1})^\top \begin{bmatrix}
        \nabla_y h_\mathcal{I}(x,y^*)^\top h_\mathcal{I}(x,y_{(\lambda^*,\boldsymbol{\alpha}}^*) \\
        0
    \end{bmatrix} \label{eqn:added-term-2}
\end{align}
% \pswt{Kai: the last step in \cref{{eqn:added-term-2}} seems to require $H$ to have $\textrm{diag}(\lambda^*_\mathcal{I})$, not just $\textrm{diag}(\lambda^*)$, to enable the cancellation. } \kai{I think it was fixed at some point. Can you check again?}
% where the first equality is due to the top-left entry in $H$, i.e., $\nabla^2_{yy} g + (\lambda^*)^\top \nabla_{yy}^2 h$, are positive definite due to the strong convexity. This implies that the top-left entry is invertible, and thus the top rows are non-degenerated. Therefore, the product of $H$ and its pseudoinverse $H^{-1}$ will be identity on the top rows, which preserves the product in the first equality.
    
Therefore, we can compute the difference between \cref{eqn:dfdy_dydx_expansion}, \cref{eqn:added_term}, and \cref{eqn:added-term-2} to bound \cref{eqn:df-and-added-term}, and use the fact that $\nabla_y g(x,y^*) + (\lambda^*)^\top \nabla_y h(x,y^*) = 0$: 
% \pswt{Since we are currently bounding \cref{eqn:df-and-added-term}, don't we also need to include the term \[- \alpha_2 \begin{bmatrix}
        % 0 \\
        % \text{diag}(1/\lambda^*) h(x,y_{\lambda^*,\boldsymbol{\alpha}}^*) 
    % \end{bmatrix} ^\top M M^{-1}   \begin{bmatrix}
        % \nabla^2_{yx} g + (\lambda^*)^\top \nabla_{yx}^2 h \\
        % \text{diag}(\lambda^*) \nabla_x h 
    % \end{bmatrix} = - \alpha_2 \begin{bmatrix}
            % h(x,y_{\lambda^*,\boldsymbol{\alpha}}^*)^\top \nabla_y h(x,y^*) \\
            % 0
        % \end{bmatrix}^\top M^{-1}   \begin{bmatrix}
        % \nabla^2_{yx} g + (\lambda^*)^\top \nabla_{yx}^2 h \\
        % \text{diag}(\lambda^*) \nabla_x h 
    % \end{bmatrix}\]?   Computing the inner product, this simplifies to \[\alpha_2 \cdot h(x, y^*_{\alpha})_{+} \cdot \|\nabla_x h(x, y^*)\|.\] I'm currently not able to see if it's being bounded somewhere. Note that a very similar-looking term appeared later when bounding \cref{eqn:h2-difference}, where we seem to have used $h(x, y^*_{\alpha})_{+}^\top \nabla_y h(x, y^*)=0$, so perhaps we could use a similar approach for this term, if needed?} \pswt{<- added term} 
\begin{align}
    \footnotesize
    & \frac{d y^*}{d x}^\top \nabla_y 
 f(x,y_{\lambda^*,\boldsymbol{\alpha}}^*) - \text{added term 1 } - \text{added term 2}  \nonumber \\
    = & \begin{bmatrix}
    \nabla^2_{yx} g + (\lambda^*)^\top \nabla_{yx}^2 h \\
        \text{diag}(\lambda^*_\mathcal{I}) \nabla_x h_\mathcal{I}
    \end{bmatrix}^\top (H^{-1})^\top \biggr( \alpha_1 \begin{bmatrix}
        \nabla_y g(x,y_{\lambda^*,\boldsymbol{\alpha}}^*) - \nabla_y g(x,y^*) - \nabla^2_{yy} g(x,y^*) (y_{\lambda^*,\boldsymbol{\alpha}}^* - y^*) \\ 
        0
        \end{bmatrix}\label{eqn:g_second_order_difference} \\
     & +
    \alpha_1 \begin{bmatrix}
        \nabla_y h(x,y_{\lambda^*,\boldsymbol{\alpha}}^*)^\top \lambda^* - \nabla_y h(x,y^*)^\top \lambda^* - \nabla^2_{yy} h(x,y^*)^\top \lambda^* (y_{\lambda^*,\boldsymbol{\alpha}}^* - y^*) \\
        0
    \end{bmatrix}\label{eqn:h_second_order_difference}
    \\
    & \quad -
    \alpha_1 \begin{bmatrix}
        0 \\ 
        \nabla_y h_\mathcal{I}(x,y^*) (y_{\lambda^*,\boldsymbol{\alpha}}^* - y^*)
    \end{bmatrix} \label{eqn:constraint_difference}
    \\
    & \quad \quad + \alpha_2  \begin{bmatrix}
        \nabla_y h_\mathcal{I}(x,y_{\lambda^*,\boldsymbol{\alpha}}^*)^\top h_\mathcal{I}(x,y_{\lambda^*,\boldsymbol{\alpha}}^*)  \\
        0 
    \end{bmatrix} - \begin{bmatrix}
        \nabla_y h_\mathcal{I}(x,y^*)^\top h_\mathcal{I}(x,y_{\lambda^*,\boldsymbol{\alpha}}^*) \\
        0
    \end{bmatrix} \biggr)
    \label{eqn:h2-difference}
\end{align}
% \pswt{A factor of $\textrm{diag}(\lambda^*)$ in \cref{{eqn:constraint_difference}}? }\kai{I think this was resolved. It was just due to the matrix transpose.} 
The terms in \cref{eqn:g_second_order_difference} and \cref{eqn:h_second_order_difference} can both be bounded by $\alpha_1 C_{g} L_y \|y_{\lambda^*,\boldsymbol{\alpha}}^* - y^*\|^2$ and $\alpha_1 R C_{h} L_y \|y_{\lambda^*,\boldsymbol{\alpha}}^* - y^*\|^2$ by the smoothness of $g$ and $h^\top \lambda^*$. Further, plugging in   $\|y^*- y_{\lambda^*,\boldsymbol{\alpha}}^*\| \leq O({\alpha_1^{-1}})$ from \cref{thm:solution-bound} bounds both these terms by $O({\alpha_1^{-1}})$. 
% \pswt{Kai: are we perhaps missing an $L_y$ factor in the two bounds here since we also have $\|dy^*/dx\|$?} \kai{Yes you are right! Thank you!}


To bound the term in \cref{eqn:constraint_difference}, we use:
\begin{align*}%\label{eqn:h-smoothness}
    \norm{h_\mathcal{I}(x,y_{\lambda^*,\boldsymbol{\alpha}}^*) - h_\mathcal{I}(x,y^*) - \nabla_y h_\mathcal{I}(x,y^*) (y_{\lambda^*,\boldsymbol{\alpha}}^* - y^*)} \leq C_h \norm{y_{\lambda^*,\boldsymbol{\alpha}}^* - y^*}^2. 
\end{align*}
Therefore, we have: 
\begin{align*}
    % \nabla_y h_\mathcal{I}(x,y^*) (y_{\lambda^*,\boldsymbol{\alpha}}^* - y^*) & = h_\mathcal{I}(x,y_{\lambda^*,\boldsymbol{\alpha}}^*) - h_\mathcal{I}(x,y^*) + C_h O(\norm{y_{\lambda^*,\boldsymbol{\alpha}}^* - y^*}^2)  \\
     \norm{\nabla_y h_\mathcal{I}(x,y^*) (y_{\lambda^*,\boldsymbol{\alpha}}^* - y^*)} & \leq  \norm{h_\mathcal{I}(x,y_{\lambda^*,\boldsymbol{\alpha}}^*)} +\norm{h_\mathcal{I}(x,y^*)} + C_h O(\norm{y_{\lambda^*,\boldsymbol{\alpha}}^* - y^*}^2) \\
    & \leq O({{\alpha_1^{-1/2} \alpha_2^{-1/2}}}) + 0 + O({\alpha_1^{-2}}) \\
    & = O({{\alpha_1^{-1/2} \alpha_2^{-1/2}}} + {\alpha_1^{-2}}),%\numberthis\label{eqn:gamma-constraint-violation}
\end{align*}
which upon scaling by $\alpha_1$ gives us the following bound on the term in \cref{eqn:constraint_difference}:
\begin{align*}
    \alpha_1 \norm{\nabla_y h_\mathcal{I}(x,y^*) (y_{\lambda^*,\boldsymbol{\alpha}}^* - y^*)} \leq O(\alpha_1^{1/2} \alpha_2^{-1/2} + \alpha_1^{-1}) ~.
\end{align*}


The term in \cref{eqn:h2-difference} can be bounded by:
% \pswt{What is the justification for the first step? (It appears that this follows from $h(x,y_{\lambda^*,\boldsymbol{\alpha}}^*)_+^\top \nabla_y h(x,y^*)=0$, but I don't yet see why this is true)} \kai{Good point! I spent a while thinking about this and found that I need to modify the Lagrangian minimization problem with the change of $h_+$ to $h_\mathcal{I}$ instead. Please verify if there is any other issue.}
\begin{align*}
    & \alpha_2 \norm{\nabla_x h_\mathcal{I}(x,y_{\lambda^*,\boldsymbol{\alpha}}^*)^\top h_\mathcal{I}(x,y_{\lambda^*,\boldsymbol{\alpha}}^*) - \nabla_x h_\mathcal{I}(x,y^*)^\top h_\mathcal{I}(x,y_{\lambda^*,\boldsymbol{\alpha}}^*)} \\
    = & \alpha_2 \norm{\nabla_x h_\mathcal{I}(x,y_{\lambda^*,\boldsymbol{\alpha}}^*) - \nabla_x h_\mathcal{I}(x,y^*)} O(\norm{h_\mathcal{I}(x,y^*_{\boldsymbol{\alpha},\lambda^*})})  \\
    = & \alpha_2\cdot O({\alpha_1^{-1}}) O({{\alpha_1^{-1/2} \alpha_2^{-1/2}}}) \\
    = & O(\alpha_1^{-3/2} \alpha_2^{1/2}) \numberthis\label{eq:bound_on_h_plus_grady_h}
\end{align*}
% \pswt{Kai: just want to check --- in the second equation above, we are using that $h_\mathcal{I}$ is also smooth, right? Does that follow from $h$ being smooth?} \kai{Yes, it follows by $h_\mathcal{I}$ being a subset of the entries of $h$, where $h$ is convex (for each entry).}

\noindent\textbf{Bounding \cref{eqn:dgdh-and-added-term}:}
This can be easily bounded by the smoothness of $g$ and $h$, and the bound on the dual solution $\norm{\lambda^*} \leq R$.
Thus \cref{eqn:dgdh-and-added-term} can be bounded by $R \cdot O({\alpha_1^{-1}}) = O({\alpha_1^{-1}})$.
% \pswt{Kai: just wanted to check, we are keeping $R$ but seem to be putting the smoothness constants in the big-oh notation; is that ok? (perhaps this is a minor question since the final bound doesn't anyway show the constants explicitly)} \kai{added $O(\alpha^{-1})$ right after.}
% \pswt{Isn't there a $\alpha_1$ scaling factor?}

\noindent\textbf{Bounding \cref{eqn:dh2-difference}:}
By the same argument in \cref{eq:bound_on_h_plus_grady_h}, we get:
\begin{align}
    & \alpha_2 \norm{\nabla_y h_\mathcal{I}(x,y_{\lambda^*,\boldsymbol{\alpha}}^*)^\top h_\mathcal{I}(x,y_{\lambda^*,\boldsymbol{\alpha}}^*) - \nabla_y h_\mathcal{I}(x,y^*)^\top h_\mathcal{I}(x,y_{\lambda^*,\boldsymbol{\alpha}}^*) } \nonumber \\
    \leq & \alpha_2 \norm{\nabla_y h_\mathcal{I}(x,y_{\lambda^*,\boldsymbol{\alpha}}^*) - \nabla_y h_\mathcal{I}(x,y^*)} \norm{h_\mathcal{I}(x,y_{\lambda^*,\boldsymbol{\alpha}}^*)} \nonumber \\
    = &  \alpha_2\cdot O({\alpha_1^{-1}}) O({{\alpha_1^{-1/2} \alpha_2^{-1/2}}}) \nonumber \\
    = & O(\alpha_1^{-3/2}\alpha_2^{1/2}) ~. \nonumber
\end{align}

Combining all upper bounds gives the claimed bound. 
% :
% \begin{align}\label{eqn:gradient-approximation}
%    & \norm{\nabla_x F(x) - \nabla_x L_{\boldsymbol{\alpha}, \lambda^*}(x, y_{\lambda^*,\boldsymbol{\alpha}}^*)} \leq O(\frac{1}{\alpha_1}) + O(\frac{1}{\alpha_1^{1/2}\alpha_2^{1/2}}) + O(\frac{\alpha_1^{1/2}}{\alpha_2^{1/2}}) + O(\frac{\alpha_2^{1/2}}{\alpha_1^{3/2}})
% \end{align}
\end{proof}

\section{Proof of the main result (\cref{thm:cost_of_computing_ystar_gammastar_inequality}): convergence and computation cost}\label{appendix:cost_of_computing_ystar_gammastar_inequality}
\computationCostInequality*
\begin{proof}
    First, given the bound in \cref{thm:diff_in_hypergrad_and_gradLagr}, we choose $\alpha_1 = \alpha^{-2}$ and $\alpha_2 = \alpha^{-4}$ to ensure the inexactness of the gradient oracle is bounded by $\alpha$. In the later analysis, we will still use $\alpha_1$ and $\alpha_2$ in the penalty function for clarity.

    Now we estimate the computation cost of the inexact gradient oracle:
    
    \noindent\textbf{Lower-level problem.}
    Given the oracle access to the optimal dual solution $\lambda^*(x)$, we can recover the primal solution $y^*(x)$ efficiently (e.g, by \cite{zhang2022solving}). Therefore, we can use the primal and dual solutions to construct the penalty function $\mathcal{L}_{\lambda^*, \alpha}(x,y)$ in ~\cref{eqn:penalty-lagrangian}.
    
    \noindent\textbf{Penalty function minimization problem.}
    The second main optimization problem is the penalty minimization problem in \cref{line:lagrangian-optimization} of \cref{alg:inexact-gradient-oracle}.
    Recall from \cref{eqn:penalty-lagrangian} that 
    \begin{align}\label{eqn:penalty-lagrangian-approximate}
        \mathcal{L}_{\lambda,\boldsymbol{\alpha}}(x,y) = f(x,y) + \alpha_1 \left( g(x,y) + \lambda^\top h(x,y) - g^*(x)  \right) + \frac{\alpha_2}{2} \norm{h_\mathcal{I}(x,y)}^2, 
    \end{align}
    where we use the approximate dual solution $\lambda$ as opposed to the optimal dual solution $\lambda^*$.
    % \begin{remark}
    %     The uniqueness of the optimal dual solution is guaranteed under the LICQ assumption in ~\cref{assumption:linEq_smoothness}) based on the result from ~\cite{wachsmuth2013licq}.
    % \end{remark}

    Given \cref{eqn:penalty-lagrangian-approximate}, we solve the penalty minimization problem: 
    \begin{align*}
        y'_{\lambda, \alpha}(x) \coloneqq \arg\min_y \mathcal{L}_{\lambda,\boldsymbol{\alpha}}(x,y). 
    \end{align*}
    The penalty minimization is a unconstrained strongly convex optimization problem, which is known to have linear convergence rate. We further analyze its convexity and smoothness below to precisely estimate the computation cost:
    \begin{itemize}
        \item The strong convexity of $\mathcal{L}_{\lambda,\boldsymbol{\alpha}}(x,y)$ is lower bounded by $\frac{\alpha_1 \mu_g}{2} = O(\alpha_1)$.
        \item The smoothness of $\mathcal{L}_{\lambda,\boldsymbol{\alpha}}(x,y)$ is dominated by the smoothness of $\alpha_2 \norm{h_\mathcal{I}(x,y)}^2$ since $\alpha_2 \gg \alpha_1$. By \cref{thm:diff_in_hypergrad_and_gradLagr}, we know that the optimal solution must lie in an open ball $B(y^*, O(1/\alpha_1))$ with center $y^*$ (inner optimization primal solution) and a radius of the order of $O(\frac{1}{\alpha_1})$. This implies that we just need to search over a bounded feasible set of $y$, which we can bound $\norm{\nabla_y h(x,y)} \leq L_h$ and $h(x,y) \leq H$ within the bounded region $y \in B(y^*, O(1/\alpha_1))$. We can show that $h^2$ is smooth (gradient Lipschitz) within the bounded region by the following:
        \begin{align}
            \norm{\nabla^2_{yy} h^2} = \norm{h \nabla^2_{yy} h + \nabla_y h^\top \nabla_y h} \leq \norm{h \nabla^2_{yy} h} + \norm{\nabla_y h^\top \nabla_y h} \leq H C_h + L_h^2 \nonumber
        \end{align}
        which also implies $h^2_{\mathcal{I}}$ is also smooth (gradient Lipschitz).
        Therefore, $\alpha h^2_{\mathcal{I}}$ is $( H C_h + L_h^2) \alpha_2 = O(\alpha_2)$ smooth.
    \end{itemize}
     
    Choosing $\alpha_1 = \frac{1}{\alpha^2}$ and $\alpha_2 = \frac{1}{\alpha^4}$, the condition number of $\mathcal{L}_{\boldsymbol{\alpha}, \lambda}(x,y)$ becomes $\kappa = O(\alpha_2 / \alpha_1) = O(\frac{1}{\alpha^2})$.
    Therefore, by the linear convergence of gradient descent in strongly convex smooth optimization, the number of iterations needed to get to $\alpha$ accuracy is $O(\sqrt{\alpha^{-2}} \times \log (\frac{1}{\alpha})) = O(\frac{1}{\alpha} \log (\frac{1}{\alpha}))$.
    Therefore, we can get a near optimal solution $y'_{\lambda,\alpha}$ with inexactness $\alpha$ in $O(\frac{1}{\alpha})$ oracle calls.

\noindent\textbf{Computation cost and results.}
Overall, for the inner optimization, we can invoke the efficient optimal dual solution oracle to get the optimal dual solution $\lambda^*(x)$ and recover the optimal primal solution $y^*(x)$ from there.
For the penalty minimization, we need $O(\frac{1}{\alpha})$ oracle calls to solve an unconstrained strongly convex smooth optimization problem to get to $\alpha$ accuracy.
In conclusion, combining everything in \cref{appendix:cost_of_computing_ystar_gammastar_inequality}, we run $O(\frac{1}{\alpha})$ oracle calls to obtain an $\alpha$ accurate gradient oracle to approximate the hyperobjective gradient $\nabla_x F(x)$. This concludes the proof of \cref{thm:cost_of_computing_ystar_gammastar_inequality}.
\end{proof}

\begin{remark}
    The following analysis  quantifies how the error in the optimal dual solution propagates to the inexact gradient estimate. This is not needed if such a dual solution oracle exists. But in practice, the oracle may come with some error, for which we bound the error.
\end{remark}

\noindent\textbf{Bounding the error propagation in error in dual solution and the penalty minimization.}
First, if we do not get an exact optimal dual solution, the error in the dual solution $\lambda$ with $\norm{\lambda - \lambda^*} \leq \alpha$ will slightly impact the analysis in \cref{thm:diff_in_hypergrad_and_gradLagr}. Specifically, in \cref{appendix:proof-of-inexact-gradient}, the approximate $\lambda$ will impact the inexact gradient $\nabla_x \mathcal{L}_{\lambda,\alpha}(x,y'_{\lambda,\alpha})$ computation and the analysis in \cref{eqn:dgdh-and-added-term} and \cref{eqn:y_lambda_star_optimality}.
% \begin{itemize}
    % \item 
    In \cref{eqn:dgdh-and-added-term}, to change $\lambda$ to $\lambda^*$, we get an additional error:
    \begin{align*}\numberthis\label{eqn:dual-convergence-merit-function-1}
        & \alpha_1 \biggl( \nabla_x h(x,y')^\top (\lambda - \lambda^*) - \nabla_x h(x,y'_{\lambda,\alpha})^\top (\lambda - \lambda^*) \biggl) \nonumber \\
        = & \alpha_1 (\nabla_x h(x,y') - \nabla_x h(x,y'_{\lambda,\alpha}))^\top (\lambda - \lambda^*) \nonumber \\
        \leq & \alpha_1 C_h \norm{y' - y'_{\lambda,\alpha}} (\lambda - \lambda^*) \nonumber \\
        \leq & O(\alpha_1 \alpha_1^{-1} \alpha) = O(\alpha), 
    \end{align*}
    where the last inequality is due to $\norm{y' - y'_{\lambda,\alpha}} \leq O(\alpha_1^{-1})$ that is based on a similar analysis in \cref{thm:solution-bound} with a near-optimal $y'_{\lambda,\alpha}$ under $\alpha^2 = \alpha_1$ accuracy.
    
    Therefore, the error incurred by inexact $\lambda$ in \cref{eqn:dgdh-and-added-term} is at most $O(\alpha)$, which is of the same rate as the current gradient inexactness $O(\alpha)$.
    % \item 
    
    In \cref{eqn:y_lambda_star_optimality}, the optimality holds approximately for the approximate $\lambda$. Therefore, by the near optimality of $y_{\lambda,\boldsymbol{\alpha}}'$ (strongly convex optimization), we know that the following gradient is also $\alpha$-close to $0$, i.e.,
    \begin{align*}\numberthis\label{eqn:inexact-optimality-with-inexact-lambda}
        & \|\nabla_y f(x,y_{\lambda,\boldsymbol{\alpha}}') + \alpha_1 \left( \nabla_y g(x,y_{\lambda,\boldsymbol{\alpha}}') + \nabla_y h(x,y_{\lambda,\boldsymbol{\alpha}}')^\top \lambda \right) \\
        &\quad + \alpha_2 \nabla_y h_\mathcal{I}(x,y_{\lambda,\boldsymbol{\alpha}}')^\top h_\mathcal{I}(x,y_{\lambda,\boldsymbol{\alpha}}')\| \leq \alpha, 
    \end{align*}
    whose inexactness matches the inexactness of the gradient oracle $\alpha$, and thus we do not incur additional order of inexactness here.
    
    Moreover, there is an additional error because we need $\lambda^*$ as opposed to a near-optimal $\lambda$ to make the analysis in \cref{appendix:proof-of-inexact-gradient} work. The error between using $\lambda$ and $\lambda^*$ in \cref{eqn:inexact-optimality-with-inexact-lambda} can be bounded by:
    \begin{align}\label{eqn:dual-convergence-merit-function-2}
        \norm{\nabla_y h(x,y'_{\lambda,\alpha})^\top (\lambda - \lambda^*)} \leq L_h \alpha, 
    \end{align}
    where we use the local Lipschitzness of the function $h$ in an open ball near $y^*$.
    Therefore, the additional error is also  $O(\alpha)$, which matches the inexactness of the inexact gradient oracle. 
% \end{itemize}

Therefore, we conclude that in order to bound the inexactness of the gradient oracle, we just need an efficient inexact dual solution with $\alpha$ accuracy.


\section{Practical oracle to optimal (approximate) dual solution}
Here we discuss how practical the assumption on the oracle access to the optimal dual solution is.

For linear inequality constraint $h(x,y) = Ax - By - b$, the LL problem is a constrained strongly convex smooth optimization problem. 
    % All we need from the LL strongly convex smooth optimization problem is a convergence guarantee for the dual variable.
    % More precisely, we just need the dual variable $(\lambda^*)^t$ to be $1/t$-close to the optimal dual solution $\lambda^*$ in $t$ iterations.
    % In other word, we expect to find a dual solution $\lambda^t$ to the inner optimization problem with $\norm{\lambda^t - \lambda^*} \leq \alpha $ in $t = O(1/\alpha)$ iterations.
To show that we can compute an approximate solution to the optimal dual solution for linear inequality constraints, we apply the result from ~\cite{du2019linear}:
\begin{corollary}[Application of Corollary 3.1 in \cite{du2019linear}]\label{cor:linear-convergence-linear-inequality-LL}
    When $h(x,y) = Ax - By + b$ is linear in $y$, the primal and dual solutions can be written as:
    \begin{align}
        & y^*, \lambda^* = \arg\min_y \max_\lambda g(x,y) + (\lambda^*)^\top h(x,y) = g(x,y) - (\lambda^*)^\top B y + R(x) \nonumber \\
        \iff & y^*, \lambda^* = \arg\min_y \max_\lambda g(x,y) - (\lambda^*)^\top B y
    \end{align}
    where $g$ is strongly convex in $y$ and $B$ is of full rank by \cref{assumption:linEq_smoothness}. According to Corollary 3.1 from \cite{du2019linear}, the primal-dual gradient method guarantees a linear convergence. More precisely, in $t = O(\log \frac{1}{\alpha})$ iterations, we get:
    \begin{align}
        \norm{y^t - y^*} \leq \alpha \text{ and } \norm{\lambda^t - \lambda^*} \leq \alpha. 
    \end{align}
\end{corollary}
Given \cref{cor:linear-convergence-linear-inequality-LL}, we can efficiently approximate the primal and dual solutions up to high accuracy with $O(\log \frac{1}{\alpha})$ oracle calls when the inequality constraints are linear. This gives us an efficient approximate oracle access to the dual solution.

\begin{remark}
    Under the assumption of an optimal dual solution oracle, all the analyses mentioned in ~\cref{sec:inequality-bilevel} hold for the general convex inequality constraints. However, the main technical challenge is that the dual solution oracle for general convex inequality cannot be guaranteed in practice.
    In fact, to the best of our knowledge, there is no iterate convergence in the dual solution $\lambda$ for general convex inequality constraints. Most of the literature in strongly-convex-concave saddle point convergence only guarantees dual solution convergence in terms of its duality gap or some other merit functions. We are not aware of any successful bound on the dual solution iterate convergence, which is an important research question to answer by itself.
    This is the main technical bottleneck for general convex inequality constraints as well.
\end{remark}
\begin{remark}
    On the other hand, we need the dual solution iterate convergence with rate $O(1/\alpha)$ to ensure the error to be bounded. But this is not a necessary condition. To ensure a bound on the error propagation, we just need to bound some forms of merit functions (\cref{eqn:dual-convergence-merit-function-1} and \cref{eqn:dual-convergence-merit-function-2}) of the dual solutions, which we believe that this is much more tractable than the actual iterate dual solution convergence. We leave this as a future direction and this will generalize the analysis from linear inequality constraints to general convex inequality constraints.
\end{remark}

% \section{Numerical precision issue in finding active constraint sets}\label{appendix:misspecified-active-constraint-set}
% In \cref{sec:inequality-bilevel}, we assume access to the active constraint set $\mathcal{I} \coloneqq \{i \in [d_h]: h_i(x,y^*) = 0, \lambda_i > 0 \}$. However, in practice, we may not be able to find the exact active constraint set due to precision issue.
% Even though in the linear inequality case where we have linear convergence in both the primal and dual solutions, we still may mis-specify nearly-active constraints as long as their violation is small.
% In practice, suppose we run the primal and dual solutions up to $O(\epsilon)$ accuracy, we can select an $\epsilon$-approximate active constraint set by choosing $\mathcal{I}_{\text{approx}} = \{i \in [d_h]: \norm{h_i(x,y^*)} < \epsilon, \lambda_i > \epsilon \}$.
% Here, we discuss how to select the accuracy $\epsilon$ for the primal and dual solutions to ensure the error incurred by incorrect active constraint set can be bounded.

% First, this analysis is based on assuming that the eigenvalues of the KKT matrix $H(x)$ (defined in \cref{eqn:kkt-system} for a given $x$) with the corresponding accurate active constraint set $\mathcal{I}$ is lower bounded by $\epsilon_H$ and upper bounded by $C_H$.
% Since $H$ is invertible due to the linear independence of $\nabla_y h_\mathcal{I}$ with positive dual variable $\lambda_\mathcal{I} > 0$, we know that $H$ is invertible and thus its smallest eigenvalue is positive too.
% This assumption on the eigenvalue lower bound may not hold in general, but if we have bounded feasible region $x \in \mathcal{X}$ with LICQ assumption (\cref{item:assumption_safe_constraints}), we know that the KKT matrix is lower bounded due to the invertibility of $H$ and the feasible region $\mathcal{X}$ is compact.
% So for compact upper level feasible set, the assumption automatically holds.


% Now we can formally bound the error incurred by mis-specifying the set $\mathcal{I}_{\text{incor}}$ as active constraints (note that we will correctly identify the active constraints). 
% The incorrectly specified set must have $\norm{h_i(x,y')} \leq \epsilon$ and $\lambda_i > \epsilon$.

% Now we can write the incorrectly specified KKT system in \cref{eqn:kkt-system} by:
% \begin{align}\label{eqn:kkt-system-extended}
% \begin{bmatrix}
% \nabla^2_{yy} g + (\lambda)^\top \nabla_{yy}^2 h & \nabla_y h_\mathcal{I}^\top & \nabla_y h_{\mathcal{I}_{\text{incor}}}^\top \\
% \text{diag}(\lambda_\mathcal{I}) \nabla_y h_\mathcal{I} & 0 & 0 \\
% \text{diag}(\lambda_{\mathcal{I}_{\text{incor}}}) \nabla_y h_{\mathcal{I}_{\text{incor}}} & 0 & 0
% \end{bmatrix}
% \begin{bmatrix}
%     \frac{d y(x)}{d x} \\
%     \frac{d \lambda_\mathcal{I}(x)}{d x} \\
%     \frac{d \lambda_{\mathcal{I}_{\text{incor}}}(x)}{d x}
% \end{bmatrix}
% = 
% -
% \begin{bmatrix}
%     \nabla^2_{yx} g + (\lambda)^\top \nabla_{yx}^2 h \\
%     \text{diag}(\lambda_\mathcal{I}) \nabla_x h_\mathcal{I} \\
%     \text{diag}(\lambda_{\mathcal{I}_{\text{incor}}}) \nabla_x h_{\mathcal{I}_{\text{incor}}}
% \end{bmatrix}
% \end{align}
% where everything is the same except for an additional set of constraints.

% We can write the KKT matrix as a block matrix:
% \begin{align}\label{eqn:inexact-kkt-system}
%     \begin{bmatrix}
%         H' & h_{\mathcal{I}_{\text{incor}}}^\top \\
%         \text{diag}(\lambda_{\mathcal{I}_{\text{incor}}}) \nabla_y h_{\mathcal{I}_{\text{incor}}} & 0
%     \end{bmatrix}
%     \begin{bmatrix}
%         \frac{d [y,\lambda_\mathcal{I}]}{d x} \\
%         \frac{d \lambda_{\mathcal{I}_{\text{incor}}}(x)}{d x}
%     \end{bmatrix}
%     = 
%     - 
%     \begin{bmatrix}
%         b \\
%         \text{diag}(\lambda_{\mathcal{I}_{\text{incor}}}) \nabla_x h_{\mathcal{I}_{\text{incor}}}
%     \end{bmatrix}
% \end{align}
% where $b' = 
% \begin{bmatrix}
%     \nabla^2_{yx} g + (\lambda)^\top \nabla_{yx}^2 h \\
%     \text{diag}(\lambda_\mathcal{I}) \nabla_x h_\mathcal{I}
% \end{bmatrix}$

% We know that $H', b'$ are both $\epsilon$ accurate compared to the correct $H$ and gradients $b = 
% \begin{bmatrix}
%     \nabla^2_{yx} g + (\lambda^*)^\top \nabla_{yx}^2 h \\
%     \text{diag}(\lambda_\mathcal{I}) \nabla_x h_\mathcal{I}
% \end{bmatrix}$, and the correct gradient with the exact active set of constraints is computed by $-H^{-1} b$.
% Therefore, our goal is to bound the difference between the solution to Equation~\cref{eqn:inexact-kkt-system} and $-H^{-1} b$.

% To see this, we write the following using Schur complement:
% \begin{align}
%     & \begin{bmatrix}
%         H' & h_{\mathcal{I}_{\text{incor}}}^\top \\
%         \text{diag}(\lambda_{\mathcal{I}_{\text{incor}}}) \nabla_y h_{\mathcal{I}_{\text{incor}}} & 0
%     \end{bmatrix}^{-1} 
%     \begin{bmatrix}
%         b \\
%         \text{diag}(\lambda_{\mathcal{I}_{\text{incor}}}) \nabla_x h_{\mathcal{I}_{\text{incor}}}
%     \end{bmatrix}
%     \\
%     = & 
%     \begin{bmatrix}
%         I & - H^{-1} \nabla_y h_{\mathcal{I}_{\text{incor}}}^\top \\
%         0 & I
%     \end{bmatrix}
%     \begin{bmatrix}
%         H^{-1} & 0 \\
%         0 & - (\text{diag}(\lambda_{\mathcal{I}_{\text{incor}}}) \nabla_y h_{\mathcal{I}_{\text{incor}}} H^{-1} \nabla_y h_{\mathcal{I}_{\text{incor}}}^\top)^{-1} 
%     \end{bmatrix}  \\
%     &\quad\times
%     \begin{bmatrix}
%         I & 0 \\
%         - \text{diag}(\lambda_{\mathcal{I}_{\text{incor}}}) \nabla_y h_{\mathcal{I}_{\text{incor}}} H^{-1} & I
%     \end{bmatrix}
%     \begin{bmatrix}
%         b \\
%         \text{diag}(\lambda_{\mathcal{I}_{\text{incor}}}) \nabla_x \nabla_y h_{\mathcal{I}_{\text{incor}}}
%     \end{bmatrix} \nonumber \\
%     = & 
%     \begin{bmatrix}
%         I & - H^{-1} \nabla_y h_{\mathcal{I}_{\text{incor}}}^\top \\
%         0 & I
%     \end{bmatrix}
%     \begin{bmatrix}
%         H^{-1} & 0 \\
%         0 & - (\text{diag}(\lambda_{\mathcal{I}_{\text{incor}}}) \nabla_y h_{\mathcal{I}_{\text{incor}}} H^{-1} \nabla_y h_{\mathcal{I}_{\text{incor}}}^\top)^{-1} 
%     \end{bmatrix}
%     \begin{bmatrix}
%         b \\
%         O(\epsilon \epsilon_H^{-1})
%     \end{bmatrix} \nonumber \\
%     = & 
%     \begin{bmatrix}
%         I & - H^{-1} \nabla_y h_{\mathcal{I}_{\text{incor}}}^\top \\
%         0 & I
%     \end{bmatrix}
%     \begin{bmatrix}
%         H^{-1} b \\
%         O(\epsilon \epsilon_H^{-1}) \cdot O(\epsilon^{-1} \epsilon_H L_h^{-2})
%     \end{bmatrix} \nonumber \\
%     = & 
%     \begin{bmatrix}
%         H^{-1} b \\
%         O(L_h^{-2} \epsilon_H^{-1} )
%     \end{bmatrix} \nonumber \\
% \end{align}
% where the second to the last equality uses $\norm{\text{diag}(\lambda_{\mathcal{I}_{\text{incor}}}) h_{\mathcal{I}_{\text{incor}}} H^{-1} h_{\mathcal{I}_{\text{incor}}}^\top} \geq \epsilon C_H^{-1} \norm{\nabla_y h_{\mathcal{I}_{\text{incor}}}}^2 \geq \epsilon \epsilon_H^{-1} L_h^2$ with local $h$ Lipschitzness around $y^*$, where in the linear inequality case $h(x,y) = Ax - By - b$ we can directly compute an upper bound using $\norm{B}$.
% Thus its matrix inversion is lower bounded by $\epsilon^{-1} \epsilon_H L_h^{-2}$.





\section{The role of $\lambda^*(x)$ in the derivative of Equation~\cref{eqn:penalty-lagrangian}}
Notice that Equation~\cref{eqn:penalty-lagrangian}, we treat the dual solution $\lambda^*(x)$ as a constant to define the penalty function derivative. Yet, the dual solution $\lambda^*(x)$ is in fact also a function of $x$. Therefore, in theory, we should also compute its derivative with respect to $x$.

However, notice that the following:
\begin{align}\label{eqn:lambda_derivative}
    \nabla_x (\lambda^*(x))^\top h(x,y) & = \nabla_x h(x,y)^\top \lambda^* + \frac{d \lambda(x)}{dx}^\top h(x,y)
\end{align}
The later term in Equation~\cref{eqn:lambda_derivative} can be divided into two cases:
\begin{itemize}
    \item For active constraint $i \in \mathcal{I}$ with $h(x,y^*) = 0$, we know that $y^*_{\lambda,\alpha}$ is close to $y^*$ by \cref{thm:solution-bound}. Therefore, the derivative $\norm{\frac{d \lambda(x)}{dx}^\top h(x,y^*_{\lambda,\alpha})} \leq L_h L_\lambda \alpha_1 = O(\alpha_1) = O(\alpha^2)$ by the local smoothness of $h$ near $y^*$ and the Lipschitzness assumption of $\lambda^*$ in \cref{item:assumption_safe_constraints}.
    \item For inactive constraint $i \in \bar{\mathcal{I}}$ and $\lambda^*_i > 0$, we can solve the KKT conditions and get $\frac{d \lambda(x)}{dx} = 0$. Therefore, the second term becomes $0$.
    \item For inactive constraint $i \in \bar{\mathcal{I}}$ and $\lambda^*_i = 0$, the KKT system degenerates and we need to use subgradient. By solving the KKT system, we find that $\frac{d \lambda(x)}{dx} = 0$ is a valid subgradient. Therefore, by choosing this subgradient, the second term also vanishes.
\end{itemize}
Therefore, we do not need to compute the derivative of $\lambda^*$ as the terms involved its derivative is negligible compared to other major terms.

\input{src/appendix-experiment}